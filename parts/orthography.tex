
\newpage
\section{Orthography}
Thiguka's orthography is phonetic, and is as follows.

\begin{table}[h!]
    \centering
    \caption{Thiguka orthography}
    \begin{tabularx}{4cm}{|X|X|}
        \hline
        \textbf{IPA} & \textbf{Letter} \\
        \hline
        p & p \\
        t & t \\
        k & k \\
        b & b \\
        d & d \\
        g & g \\
        f & f \\
        s & s \\
        r & r \\
        l & l \\
        \textipa{T} & th \\
        a & a \\
        e & e \\
        i & i \\
        o & o \\
        u & u \\
        \textipa{A} & ah \\
        ai & ay \\
        ei & ey \\
        ui & uy \\
        au & aw \\
        \textipa{Ai} & auy \\
        ia & ija \\
        \hline
    \end{tabularx}
\end{table}

Thiguka follows English punctuation conventions.  
Proper nouns are capitalized, but if they are being attached to a word, the prefix that comes before the adjective is usually separated from the adjective by a hyphen: tagu-Ayris, not *Taguayris or *taguAyris.

Loanwords are respelled to fit this orthography. The native spelling can be provided after the loanword in a parenthesis, for example:

\begin{quotation}
    Lela Reyl (Ilgis: Lena Raine) gulaya tagu-Arika tarifari.
\end{quotation}


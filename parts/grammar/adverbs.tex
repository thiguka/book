\section{Adverbs}
In Thiguka, adverbs are either placed before the verb they modify or as clitics to the end of the verb.

These adverbs may be marked with the marker \textit{-se}, but can be omitted if there is sufficient context.

\ex
\begingl
\gla   Rafu-pah pa-leyle-ta=thasi rafu-pafay   balisa-lay-sa   Katrila-tay.//
\glb   3PL-NOM  PFV-name-PST=also 3PL-GEN.POSS daughter-DAT-SG Katrina-ACC//
\glft  ``They also named their daughter Katrina.''//
\endgl
\xe


However, \textit{-se} is mandatory when adverbializing an adjective:

\ex
\begingl
\glpreamble Usage of \textit{free} as an adverb:// 
\gla   Thaka-pah pa-paratas-ta=   @ li-pera-      @ se      kasa kolusute-tay-sa//
\glb   1SG-NOM   PFV-receive-PST=   NEG-money.ADV   -ADVZ   this computer-ACC-SG//
\glft  I received this computer for free.//
\endgl
\xe

\section{Comparatives}
There are two primary comparatives in Thiguka, \textit{bas} ("more") and \textit{las} ("less").

\subsection{Quantity}
To indicate a greater quantity of something and a quantity of some other thing lesser than the greater quantity, the two are placed on both sides of the word.
The greater quantity noun is to the left, while the lesser quantity noun is to the right.

\ex
\begingl
\gla Kake-pah ge-pafay-sa fahfalusaya-elah bas  fahlifalusayaelah//  
\glb 2SG-NOM  PROG-has-SG fan-PL           more hater-PL//  
\glft You have more fans than haters.//  
\endgl
\xe

\ex
\begingl
\gla Kake-pah ge-pafay-sa fahlifalusaya-elah bas  fahfalusayaelah//  
\glb 2SG-NOM  PROG-has-SG hater-PL           more fan-PL//  
\glft You have more haters than fans.//  
\endgl
\xe

\subsection{Adjectives}
For adjectives, the comparatives are attached as clitics.

\ex
\begingl
\gla Kake-pafay ka\~{}gu-gahlirithi kala-pah      gulaya      gu-gulid=bas  thaka-pafay kala.//  
\glb 2SG-GEN.POSS \agradj{}-violin  knowledge-NOM COP.3SG.PRS AGR-good=more 1SG-GEN.POSS knowledge //  
\glft Your knowledge of the violin is better than my knowledge.//  
\endgl
\xe




\section{Case}
Thiguka features six cases, with their own specific uses.
Case marking is completely optional.
A completely uninflected word is that word's dictionary form.

\begin{table}[h!]
    \centering
    \caption{Cases}
    \begin{tabularx}{8cm}{|X|X|}
        \hline
        \textbf{Case} & \textbf{Suffix} \\
        \hline
        Nominative & -pah \\
        Accusative & -tay \\
        Dative & -lay \\
        Genitive & -pafay \\
        Locative & -kala \\
        Instrumental & -kithi \\ 
        \hline
    \end{tabularx}
\end{table}

\subsection{Nominative}
Nominative case is used in Thiguka to mark subjects.

\ex
\begingl
\gla Thaka-\textbf{pah} falusaya rifari-tay-elah//
\glb 1SG-\textbf{NOM}   like     music-ACC-PL//
\glft ``I like music.''//
\endgl
\xe

\subsection{Accusative}
Accusative case is used in Thiguka to mark objects.

\ex
\begingl
\gla Thaka-pah falusaya rifari-\textbf{tay}-elah//
\glb 1SG-NOM   like     music-\textbf{ACC}-PL//
\glft ``I like music.''//
\endgl
\xe

\subsection{Dative}
Dative case is used in Thiguka for the beneficiary of an action.

\ex
\begingl
\gla Thaka-pah pa-kalath-a   Thiguka-tay kake-\textbf{lay}//
\glb 1SG-NOM   PFV-teach-FUT Thiguka-ACC 2SG-\textbf{DAT}//
\glft``I will teach you Thiguka.''//
\endgl
\xe
Both \textit{kalaththa} and \textit{kalatha} are valid future tense forms of \textit{kalath} (``to teach'').

\subsection{Genitive}
Genitive case has only been used by Lemuria in his Thiguka writings, to indicate posession.

\ex
\begingl
\gla  Kasa-pah     gulaya     Gahlija-\textbf{pafay} gahlirithi.//
\glb  PROX.DET-NOM COP.3SG.PRS Gallia-\textbf{GEN}   violin//
\glft ``This is Gallia's violin.''//
\endgl
\xe



\subsection{Locative}
Locative case is used for a location or a time something happened.
It can be combined with the preposition \textit{ifil}.

\ex
\begingl
\gla  Pothu-pah pa-teligara-ta        ifil Los Algeles-\textbf{kala} ifil literidilay-\textbf{kala}//
\glb  3SG-NOM   PFV-twitch.stream-PST in   Los Angeles-\textbf{LOC}  in   yesterday-\textbf{LOC}//
\glft ``She streamed on Twitch in Los Angeles yesterday.'' (lit. ``She streamed on Twitch in Los Angeles, in yesterday'')//
\endgl
\xe

\subsection{Instrumental}
Instrumental case is used to denote, in general, the methods through which the subject did something.

\ex
\begingl
\gla  Pothu-pah pa-sapakalakesi-ta ka\~{}gu-rifari  kalakesi Lipset-\textbf{kithi} ifil Los Algeles-kala, Kaliforlija-kala.//
\glb  3SG-NOM   PFV-graduate-PST   \agradj{}-music  school   Lipsett-\textbf{INS}  in   Los Angeles-LOC   California-LOC.//
\glft ``She graduated from the music school under Lipsett in Los Angeles, California.''//
\endgl
\xe
In this example, perhaps Lipsett had been ``her'' primary music teacher the whole time she was at the school.



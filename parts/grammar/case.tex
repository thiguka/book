\newpage{}

\subsection{Case}
Thiguka features six cases, with their own specific uses.
Case marking is completely optional.
A completely uninflected word is that word's dictionary form.

\begin{table}[h!]
    \centering
    \caption{Cases}
    \begin{tabularx}{8cm}{|X|X|}
        \hline
        \textbf{Case} & \textbf{Suffix} \\
        \hline
        Nominative & -pah \\
        Accusative & -tay \\
        Dative & -lay \\
        Genitive & -pafay \\
        Locative & -kala \\
        Instrumental & -kithi \\ 
        \hline
    \end{tabularx}
\end{table}

\subsubsection{Nominative}
Nominative case is used in Thiguka to mark subjects.
\begin{exe}
    \ex{} \gll{}Thaka-\textbf{pah} falusaya rifari-tay-elah\\
                1SG-\textbf{NOM}            like     music-ACC-PL\\
        \glt{}``I will teach you Thiguka.''
\end{exe}

\subsubsection{Accusative}
Accusative case is used in Thiguka to mark objects.
\begin{exe}
    \ex{} \gll{}Thaka-pah falusaya rifari-\textbf{tay}-elah\\
                1SG-NOM            like     music-\textbf{ACC}-PL\\
        \glt{}``I will teach you Thiguka.''
\end{exe}

\subsubsection{Dative}
Dative case is used in Thiguka for the beneficiary of an action.
\begin{exe}
    \ex{} \gll{}Thaka-pah pa-kalath-a   Thiguka-tay kake-\textbf{lay}\\
                1SG-NOM   PFV-teach-FUT Thiguka-ACC 2SG-\textbf{DAT}\\
        \glt{}``I will teach you Thiguka.''
\end{exe}
Both \textit{kalaththa} and \textit{kalatha} are valid future tense forms of \textit{kalath} (``to teach'').

\subsubsection{Genitive}
Genitive case has only been used by Lemuria in his Thiguka writings, to indicate posession.

\begin{exe}
\ex{} \gll{}Kasa-pah     gulaya     Gahlija-\textbf{pafay} gahlirithi.\\
            PROX.DET-NOM COP.3SG.PRS Gallia-\textbf{GEN}   violin\\
    \glt{}``This is Gallia's violin.''
\end{exe}

\newpage

\subsubsection{Locative}
Locative case is used for a location or a time something happened.
It can be combined with the preposition \textit{ifil}.

\begin{exe}
\ex{} \gll{}Pothu-pah pa-teligara-ta        ifil Los Algeles-\textbf{kala} ifil literidilay-\textbf{kala}\\
            3SG-NOM   PFV-twitch.stream-PST in   Los Angeles-\textbf{LOC}  in   yesterday-\textbf{LOC}\\
    \glt{}``She streamed on Twitch in Los Angeles yesterday.'' (lit. ``She streamed on Twitch in Los Angeles, in yesterday'')
\end{exe}

\subsubsection{Instrumental}
Instrumental case is used to denote, in general, the methods through which the subject did something.

\begin{exe}
\ex{} \gll{}Pothu-pah pa-sapakalakesi-ta ka\~{}gu-rifari  kalakesi Lipset-\textbf{kithi} ifil Los Algeles-kala, Kaliforlija-kala.\\
            3SG-NOM   PFV-graduate-PST   \agradj{}-music  school   Lipsett-\textbf{INS}  in   Los Angeles-LOC   California-LOC.\\
    \glt{}``She graduated from the music school under Lipsett in Los Angeles, California.''
\end{exe}
In this example, perhaps Lipsett had been ``her'' primary music teacher the whole time she was at the school.

\newpage{}

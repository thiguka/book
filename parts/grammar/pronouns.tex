\section{Pronouns}
Thiguka completely lacks gender distinction in pronouns, and has a clusivity distinction in first person pronouns.

\begin{table}[h!]
    \centering
    \caption{Pronouns}
    \begin{tabularx}{15cm}{|X|X|X|X|}
        \hline
        person/number & \textbf{First} & \textbf{Second} & \textbf{Third} \\
        \hline
        singular & thaka & kake & pothu \\
        plural   & & katake & rafu \\
        inclusive plural & situ & & \\
        exclusive plural & pata & & \\
        \hline
    \end{tabularx}
\end{table}

Thiguka has no T-V distinction in pronouns. Politeness may be achieved by attaching positive adjectives to second-person pronouns, but this is discouraged and goes against Lemuria's intentions.

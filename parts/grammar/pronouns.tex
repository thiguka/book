\section{Pronouns}
Thiguka completely lacks gender distinction in pronouns, and has a clusivity distinction in first person pronouns.

\begin{table}[H]
    \centering
    \caption{Pronouns}
    \begin{tabularx}{25em}{|c|X|X|X|}
        \hline
        \multicolumn{2}{|c|}{} & \textbf{Singular} & \textbf{Plural} \\
        \hline
        \multirow{2}{*}{\textbf{First}} & \textbf{Inclusive} & \multirow{2}{*}{thaka} & situ \\
        \cline{2-2} \cline{4-4}
        & \textbf{Exclusive} & & pata \\
        \hline
        \multicolumn{2}{|c|}{\textbf{Second}} & kake & katake \\
        \hline
        \multicolumn{2}{|c|}{\textbf{Third}} & pothu & rafu \\
        \hline
    \end{tabularx}
\end{table}


Thiguka has no T-V distinction in pronouns. Politeness may be achieved by attaching positive adjectives to second-person pronouns, but this is discouraged and goes against Lemuria's intentions.

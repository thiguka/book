
\section{Tense-aspect}
Thiguka has a tense-aspect system.

Tense and aspect are optional.
If tense and aspect are omitted, the verb becomes infinitive; or the time
it took place becomes ambiguous, granting it time-independence.

\ex
\begingl
    \gla katarila//
    \glb outer.space.travel//
    \glft `to travel through outer space'//
\endgl
\xe

\ex
\begingl
    \gla ras-katarila-ta//
    \glb ACCIDENT-outer.space.travel-PRS//
    \glft `to have accidentally traveled through outer space'//
\endgl
\xe

\ex
\begingl
    \gla ge-katarila-sa//
    \glb PROG-outer.space.travel-PRS//
    \glft `to be currently traveling through outer space'//
\endgl
\xe

\begin{table}[h!]
    \centering
    \caption{Tense}
    \begin{tabularx}{8cm}{|X|X|}
        \hline
        \textbf{Tense} & \textbf{Suffix} \\
        \hline
        Past & -ta \\
        Recent past & -tu \\
        Present & -sa \\
        Future & -tha \\
        \hline
    \end{tabularx}
\end{table}

\begin{table}[h!]
    \centering
    \caption{Aspect}
    \begin{tabularx}{8cm}{|X|X|}
        \hline
        \textbf{Aspect} & \textbf{Prefix} \\
        \hline
        Perfective & pa- \\
        Perfect & fe- \\
        Progressive & ge- \\
        Imperfective & sa- \\
        Contemplative & kas- \\
        \hline
    \end{tabularx}
\end{table}

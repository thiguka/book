
\subsection{Adjectives}
Adjectives in Thiguka are placed before words.
To attach an adjective to a word, the first syllable of the modified word, followed by \emph{-gu-} should be prefixed to the adjective.

There is no enforced adjective order in Thiguka. Speakers may position adjectives as necessary to highlight particularly important qualities.

\begin{exe}
    \ex{} \gll{}ri\~{}{}gu-gulid rifari\\
    \agradj{}-good music\\
    \glt{}`good music'
\end{exe}

Adjectives can be used as nouns too:

\begin{exe}
    \ex{} \gll{}Pothu-pafay gu-gulid-pah kufa pelu-tay-elah falusaya-sa pothu-tay\\
    3SG-GEN ADJ-good-NOM allow person-ACC-PL admire-PRS 3SG-ACC\\
    \glt{}`Their greatness made them likeable.' (Their goodness made people like them.)
\end{exe}

Speakers may also choose to include additional syllables if multiple words begin with the same syllable.
\begin{exe}
    \ex{} \gll{}Fulafata-sa, pafay-sa thi\~{}gu-sapara thil-elah, alu ara\~{}gu-li-saleyfata arafat-elah alu fulahthi lores \textbf{lisu}\~{}gu-furos li-lahkela li\~{}gu-pera \textbf{lisu}re.\\
    hike-PRS, have-PRS AGR\~{}ADJ-type thing-PL, and AGR\~{}ADJ-NEG-occupation art-PL and fun.make.thing-PL COP.3PL.PRS AGR\~{}ADJ-fun NEG-make AGR\~{}ADJ-money fun.activity.\\
    \glt{}`Hiking, having themed objects, and non-professional art are fun not-money-making hobbies.'
\end{exe}    

In this example, \textit{furos} is modifying \textit{lisure}, but to use \textit{*ligufuros} would have caused ambiguity due to the presence of \textit{lilahkela} and \textit{ligupera} in the sentence.

\subsubsection{Modifying multiple words}
To use a single adjective to modify multiple words, three approaches are possible.

The first method is to use the \textit{gusi-} prefix, from the linking syllable \textit{gu} and the relativizer \textit{si}.

\begin{exe}
    \ex{} \gll{}Si-gu-isa-safi,    thaka-pah ithefali ithere-tay alu lahkela kolugeyle-tay-elah   \\
                REL-ADJ-one-th.odd 1SG-NOM   browse   internet-ACC and make    video.game-ACC-PL  \\
          \glt{}First, I browse the internet and make video games.
\end{exe}

Either syllables of all the words modified can be included, or the adjective can be duplicated for each word that needs to be modified.

If including multiple syllables of the modified words, the order of the syllables must match the order in the sentence.

(Adapted from one of Zethar's translations)
\begin{exe}
\ex{} \gll{}ka\~{}fu\~{}gu-dalata       thaka-pah kas-gasari     alu filay-fulafata\\
            AGR\~{}AGR\~{}ADJ-two 1SG-NOM   CONTEM-action  and with.cat-hike\\
      \glt{}Secondly, I practice and hike with the local cats.
\end{exe}

In this sentence, \textit{*fukagudalata} would be incorrect as it is not presenting the adjectives in the order that they are being used in the sentence.

\begin{exe}
\ex{} \gll{}ka\~{}gu-dalata thaka-pah kas-gasari alu fu\~{}gu-dalata filay-fulafata\\
            AGR\~{}ADJ-two  1SG-NOM   CONTEM-act and AGR\~{}ADJ-two  with.cat-hike\\
      \glt{}Secondly, I practice, and secondly, I hike with cats.
\end{exe}

However, the approach of using the same adjective twice will cause a semantic change in meaning.
It may simply be another thing that gets ``lost in translation''.

\begin{quote}
    It’s not clear what one needs to do for adjectives modifying more than one thing.

    --- Zethar, Ambagame3
\end{quote}
\newpage{}

\subsection{Verbs}
\subsubsection{Morpheme order}
For verbs, the order is as follows: Derivational prefixes, Negation or intensifiers, Aspect, Stem, Tense, Adverb suffixes

\subsubsection{Transitivity}
Thiguka's verbs will vary in transitivity. Early versions of Thiguka's Anki lexicon may not have sufficient transitivity tagging.

\subsubsection{Stative verbs}
Stative verbs can be expressed through a progressive verb.

A copula here like \textit{gulaya} is optional.

\begin{exe}
    \ex{} \gll{}Pothu-pah (gulaya)      ge-parakala-sa gahlirithi-tay.\\
                3SG-NOM   (COP.3SG.PRS) PROG-learn-PRS violin-ACC\\
        \glt{}``She is learning the violin.''
\end{exe}

\begin{exe}
    \ex{} \gll{}Pothu-pah ge-parakala-tha gahlirithi-tay.\\
                3SG-NOM   PROG-learn-FUT violin-ACC\\
        \glt{}``She will soon be learning the violin.''
\end{exe}

Once she's learned the violin, one can say:

\begin{exe}
    \ex{} \gll{}Pothu-pah ge-kala-sa    gahlirithi-tay.\\
                3SG-NOM   PROG-know-PRS violin-ACC\\
        \glt{}``She knows the violin.''
\end{exe}

\begin{exe}
    \ex{} \gll{}Pothu-pah ge-pafay-sa    ka\~{}gu-gahlirithi kala-tay.\\
                3SG-NOM   PROG-know-PRS \agradj{}violin knowledge-ACC\\
        \glt{}``She has knowledge of violins.''
\end{exe}


\subsection{Examples}
Some examples of verbs will follow.

parakala (``to study; to learn; to receive knowledge'')

\begin{exe}
    \ex{} \gll{}Pothu-pah ge-parakala-sa gahlirithi-tay Olifia-kithi.\\
                3SG-NOM   PROG-learn-PRS violin-ACC Olivia-INS\\
        \glt{}``She is learning the violin with Olivia.'' (transitive, +instrumental)
\end{exe}

\begin{exe}
    \ex{} \gll{}Pothu-pah ge-parakala-sa gahlirithi-tay Beti-kithi ifil ka\~{}gu-rifari kalakesi-kala ifil Sikago-kala.\\
                3SG-NOM   PROG-learn-PRS violin-ACC Betty-INS in \agradj{}-music school-LOC in Chicago-LOC\\
        \glt{}``She is learning the violin with Betty at a school in Chicago.'' (transitive, +locative, +instrumental)
\end{exe}

\begin{exe}
    \ex{} \gll{}Pothu-pah ge-parakala-sa.\\
                3SG-NOM   PROG-learn-PRS\\
        \glt{}``She is learning.'' (intransitive)
\end{exe}

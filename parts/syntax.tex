

\chapter{Syntax}
\section{Word order}
Thiguka always uses a subject-verb-object word order.
While hypothetically it may be possible to use alternate word orders due to Thiguka case marking, SVO is the intended order and alternative orders will cause possible confusion.

\section{Questions}
Two primary strategies for asking questions in Thiguka are to create declarative statements about the listener and allow the listener to fill in the blanks.

The intonation of these questions is rising; similar to English.

\ex
\begingl
    \gla  Kake-pafay leyle las//
    \glb  2SG-GEN name COP//
    \glft `Your name is?'//
\endgl
\xe


Thiguka has one single determiner, `kalisu' (what?).

A noun can be placed after `kalisu' to further form the other English wh-words.

\begin{table}[h!]
    \centering
    \caption{Translations of English wh-words}
    \begin{tabularx}{8cm}{|X|X|}
        \hline
        \textbf{Thiguka} & \textbf{English wh-word} \\
        \hline
        kalisu & what \\
        kalisu pelu & who \\
        kathisa & why \\
        kalisu pasla & where \\
        kalisu tayle & when \\
        \hline
    \end{tabularx}
\end{table}

`kathisa' is a shortened form of `kalisu thilpahelah kufa kaketay'.

\ex
\begingl
    \gla  Kalisu thil-pah-elah kufa kake-tay?//
    \glb  What thing-NOM-PL coerce 2SG-ACC?//
    \glft `What made you do it?'//
\endgl
\xe


An example of `kalisu' is as follows:

\ex
\begingl
    \gla  Kalisu las kakepafay leyle?//
    \glb  what COP 2SG-GEN name//
    \glft`What is your name?'//
\endgl
\xe


\subsection{Polar}
Polar questions are marked by the particle \emph{kalu} added to the end of a sentence.

\ex
\begingl
    \gla  Kake las pe\~{}gu-gulid pelu kalu?//
    \glb  2SG-NOM COP \agradj{} person POLAR //
    \glft`Are you a good person?'//
\endgl
\xe


To further specify what is being doubted or asked about, kalu can be moved to the end of the word being asked about in question.

\ex
\begingl
\gla  Kake las pe\~{}gu-gulid kalu pe\~{}gu-Arika pelu?//
\glb  2SG-NOM COP \agradj{}-good POLAR \agradj{}-America person //
\glft `Are you a good American person?' ``(assumes the person is American, but doubts whether they are ``good'')''//
\endgl
\xe


\section{Posession}
Posession is conveyed using the genitive case suffix \emph{-pafay}.

\ex
\begingl
    \gla  Thaka-pafay kaela-gula//
    \glb  1SG-GEN tree-DU//
    \glft `My two trees'//
\endgl
\xe


\section{Negation}
Negation is conveyed with the morpheme \emph{li}, which can either be a prefix or a particle.

Negation cannot negate negation.
Adding more negation only strengthens the negation instead of canceling it out.

\ex
\begingl
    \gla  Li! Thaka-pah li kisu-tha pothu-tay//
    \glb  No! 1SG-NOM NEG kill-FUT 3SG-ACC//
    \glft `No! I will not kill them!'//
\endgl
\xe


\ex
\begingl
    \gla  Li-thaka-pah kisu-ta pothu-tay,//
    \glb  NEG-1SG-NOM kill-PST 3SG-ACC//
    \glft `No, I didn't kill them.' lit. `Not-me killed them.'//
\endgl
\xe

\ex
\begingl
    \gla  Li-li-li-li-thaka-pah kisu-ta pothu-tay,//
    \glb  NEG-NEG-NEG-NEG-1SG-NOM kill-PST 3SG-ACC//
    \glft `I definitely did not kill them.'//
\endgl
\xe

\section{Relative clauses}
Thiguka uses the relative marker `si', which begins a clause used as the relative clause.
It goes after the noun phrase that is being relativized.

\ex
\begingl
    \gla  Kasa     pekulugusag-tay, si  thaka-pah ri-kala, gulaya      ri-gulid.//
    \glb              PROX.DET moderator-ACC    REL 1SG-NOM   INT-know COP.3SG.PRS INT-good //
    \glft``This moderator, who I know, is good.''//
\endgl
\xe

\ex
\begingl
    \gla  Asila    pekulugusag-pah-elah pa-ithesatha-ta kasa pelu-tay-sa       si  li  ga\~{}gu-ri-palad gasari//
    \glb              DIST.DET moderator-NOM-PL     PFV-ban-PST     PROX.DET person-ACC-PL REL NEG \agradj{}-INT-bad action//
    \glft``Those moderators banned someone who did nothing wrong.''//
\endgl
\xe


\section{Adverbial clauses}
Thiguka also uses the relative marker `si' for adverbial clauses. `si' is prefixed to
multiple words to further communicate the type of adverbial clause.

Temporal clauses:
\ex
\begingl
    \gla  Thaka-pah gulaya      li-sifata ifil thaka-pafay laysa-pafay ta\~{}gu-ri-gulid  tayle//
    \glb  1SG-NOM   COP.3SG.PRS NEG-alive in   1SG-GEN     mother-GEN  \agradj{}-INT-good time//
    \glft ``I was not alive during my mother's golden age.''//
\endgl
\xe


Causal clauses:
\ex
\begingl
    \gla  Kare-pah, si-dasili  li-kala  kolusute, gulaya      li-gu-asaga   pafay situ-pafay   lahpeled-pafay saleyfata//
    \glb  Karen-NOM REL-reason NEG-know computer, COP.3SG.PRS NEG-ADJ-allow have  1PL.INCL-GEN company-GEN    occupation//
    \glft``Karen, because she does not know computers, is not allowed to have a job at our company.''//
\endgl
\xe

\section{Quantity}
To communicate quantity in Thiguka, write a quantifier before the start of a noun phrase.

\ex
\begingl
    \gla  elah gahlirithi//
    \glb  many violin//
    \glft ``many violins''//
\endgl
\xe

In this gloss, both \textit{gahlirithi} and \textit{gahlirithielah} can be used.
The \textit{-elah} is already redundant due to the quantifier being placed in front of it.

In general, plurality suffixes are optional, even without a quantifier.

\ex
\begingl
    \gla  isa  gahlirithi-elah//
    \glb  one  violin-PL//
    \glft ``*one violins''//
\endgl
\xe

Of course, mismatching plurality suffixes are still wrong.

\subsection*{Quantifier words}

Some Thiguka quantifiers for vague groups are as follows:
\begin{enumerate}
    \item elah --- many
    \item lasat --- all; every
    \item rali --- few
    \item alsaf --- half
    \item tisara --- one-third
    \item keferi --- simple majority (more than 50\%{})
\end{enumerate}

\section{Passive voice}
Thiguka is a null-subject language in passive voice:

\ex
\begingl
\gla  .        kala-sa  Liburija-lay si-dasili   pothu-pah pa-lahkela-ta Thiguka-tay. //
\glb  (people) know-PRS Lemuria-DAT  REL-because 3SG-NOM   PFV-make-PST  Thiguka-ACC //
\glft Lemuria is known because he made Thiguka. (lit. know Lemuria because he made Thiguka)//
\endgl
\xe

\ex
\begingl
\gla  .        kala-sa  Gahlija-lay si-dasili   pothu-pah gulaya      fahgahlirithi. //
\glb  (people) know-PRS Gallia-DAT  REL-because 3SG-NOM   COP.3SG.PRS violinist //
\glft lit. know Gallia because she is a violinist //
\endgl
\xe

The English approach to passive voice is acceptable too:

\ex
\begingl
\gla Liburija-pah gulaya      kala-ta  si-dasili pothu-pah pa-lahkela-ta Thiguka-tay. //
\glb Lemuria-NOM  COP.3SG.PRS know-PST REL-because 3SG-NOM PFV-make-PST  Thiguka-ACC  //
\glft Lemuria is known because he made Thiguka. //
\endgl
\xe 

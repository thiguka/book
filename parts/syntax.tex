\newpage

\section{Syntax}
\subsection{Word order}
Thiguka always uses a subject-verb-object word order.
While hypothetically it may be possible to use alternate word orders due to Thiguka case marking, SVO is the intended order and alternative orders will cause possible confusion.

\subsection{Questions}
Two primary strategies for asking questions in Thiguka are to create declarative statements about the listener and allow the listener to fill in the blanks.

The intonation of these questions is rising; similar to English.

\begin{exe}
    \ex{} \gll{}Kake-pafay leyle las\\
    2SG-GEN name COP\\
    \glt{}`Your name is?'
\end{exe}

Thiguka has one single determiner, `kalisu' (what?).

A noun can be placed after `kalisu' to further form the other English wh-words.

\begin{table}[h!]
    \centering
    \caption{Translations of English wh-words}
    \begin{tabularx}{8cm}{|X|X|}
        \hline
        \textbf{Thiguka} & \textbf{English wh-word} \\
        \hline
        kalisu & what \\
        kalisu pelu & who \\
        kathisa & why \\
        kalisu pasla & where \\
        kalisu tayle & when \\
        \hline
    \end{tabularx}
\end{table}

`kathisa' is a shortened form of `kalisu thilpahelah kufa kaketay'.

\begin{exe}
    \ex{} \gll{}Kalisu thil-pah-elah kufa kake-tay?\\
    What thing-NOM-PL coerce 2SG-ACC?\\
    \glt{}`What made you do it?'
\end{exe}

An example of `kalisu' is as follows:

\begin{exe}
    \ex{} \gll{}Kalisu las kakepafay leyle?\\
    what COP 2SG-GEN name\\
    \glt{}`What is your name?'
\end{exe}

\subsubsection{Polar}
Polar questions are marked by the particle \emph{kalu} added to the bottom of a word.

\begin{exe}
    \ex{} \gll{}Kake las pe~gu-gulid pelu kalu?\\
    2SG-NOM COP \agradj{} person POLAR \\
    \glt{}`Are you a good person?'
\end{exe}

\subsection{Posession}
Posession is conveyed using the genitive case suffix \emph{-pafay}.

\begin{exe}
    \ex{} \gll{}Thaka-pafay kaela-gula\\
    1SG-GEN tree-DU\\
    \glt{}`My two trees'
\end{exe}

\subsection{Negation}
Negation is conveyed with the morpheme \emph{li}, which can either be a prefix or a particle.

Negation cannot negate negation.
Adding more negation only strengthens the negation instead of canceling it out.

\begin{exe}
    \ex{} \gll{}Li! Thaka-pah li kisu-tha pothu-tay\\
    No! 1SG-NOM NEG kill-FUT 3SG-ACC\\
    \glt{}`No! I will not kill them!'
\end{exe}

\begin{exe}
    \ex{} \gll{}Li-thaka-pah kisu-ta pothu-tay,\\
    NEG-1SG-NOM kill-PST 3SG-ACC\\
    \glt{}`No, I didn't kill them.'
    \glt{}lit. `Not-me killed them.'
\end{exe}

\begin{exe}
    \ex{} \gll{}Li-li-li-li-thaka-pah kisu-ta pothu-tay,\\
    NEG-NEG-NEG-NEG-1SG-NOM kill-PST 3SG-ACC\\
    \glt{}`I definitely did not kill them.'
\end{exe}

\subsection{Relative clauses}
Thiguka does not yet have a relative clause system. 
Translations created before the creation of the system attached information conveyed by the relative clause as adjectives, or simply created additional declarative sentences.

\begin{exe}
    \ex{} \gll{}Rey~gu-teraseleter Reyl gulaya ta~gu-ri-gulid tarifari.\\
    AGR~ADJ-transgender Raine COP.3SG.PRS AGR~ADJ-INT-good musician\\
    \glt{}`Raine, who is transgender, is a great musician.'
    \glt{}lit. `transgender-Raine is a great musician.'
    
\end{exe}

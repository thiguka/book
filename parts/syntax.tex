\newpage

\section{Syntax}
\subsection{Word order}
Thiguka always uses a subject-verb-object word order.
While hypothetically it may be possible to use alternate word orders due to Thiguka case marking, SVO is the intended order and alternative orders will cause possible confusion.

\subsection{Questions}
Two primary strategies for asking questions in Thiguka are to create declarative statements about the listener and allow the listener to fill in the blanks.

The intonation of these questions is rising; similar to English.

\begin{exe}
    \ex{} \gll{}Kake-pafay leyle las\\
    2SG-GEN name COP\\
    \glt{}`Your name is?'
\end{exe}

Thiguka has one single determiner, `kalisu' (what?).

A noun can be placed after `kalisu' to further form the other English wh-words.

\begin{table}[h!]
    \centering
    \caption{Translations of English wh-words}
    \begin{tabularx}{8cm}{|X|X|}
        \hline
        \textbf{Thiguka} & \textbf{English wh-word} \\
        \hline
        kalisu & what \\
        kalisu pelu & who \\
        kathisa & why \\
        kalisu pasla & where \\
        kalisu tayle & when \\
        \hline
    \end{tabularx}
\end{table}

`kathisa' is a shortened form of `kalisu thilpahelah kufa kaketay'.

\begin{exe}
    \ex{} \gll{}Kalisu thil-pah-elah kufa kake-tay?\\
    What thing-NOM-PL coerce 2SG-ACC?\\
    \glt{}`What made you do it?'
\end{exe}

An example of `kalisu' is as follows:

\begin{exe}
    \ex{} \gll{}Kalisu las kakepafay leyle?\\
    what COP 2SG-GEN name\\
    \glt{}`What is your name?'
\end{exe}

\subsubsection{Polar}
Polar questions are marked by the particle \emph{kalu} added to the end of a sentence.

\begin{exe}
    \ex{} \gll{}Kake las pe~gu-gulid pelu kalu?\\
    2SG-NOM COP \agradj{} person POLAR \\
    \glt{}`Are you a good person?'
\end{exe}

To further specify what is being doubted or asked about, kalu can be moved to the end of the word being asked about in question.

\begin{exe}
    \ex{} \gll{}Kake las pe~gu-gulid kalu pe~gu-Arika pelu ?\\
    2SG-NOM COP \agradj{}-good POLAR \agradj{}-America person \\
    \glt{}`Are you a good American person?'
    \glt{}``(assumes the person is American, but doubts whether they are ``good'')''
\end{exe}

\subsection{Posession}
Posession is conveyed using the genitive case suffix \emph{-pafay}.

\begin{exe}
    \ex{} \gll{}Thaka-pafay kaela-gula\\
    1SG-GEN tree-DU\\
    \glt{}`My two trees'
\end{exe}

\subsection{Negation}
Negation is conveyed with the morpheme \emph{li}, which can either be a prefix or a particle.

Negation cannot negate negation.
Adding more negation only strengthens the negation instead of canceling it out.

\begin{exe}
    \ex{} \gll{}Li! Thaka-pah li kisu-tha pothu-tay\\
    No! 1SG-NOM NEG kill-FUT 3SG-ACC\\
    \glt{}`No! I will not kill them!'
\end{exe}

\begin{exe}
    \ex{} \gll{}Li-thaka-pah kisu-ta pothu-tay,\\
    NEG-1SG-NOM kill-PST 3SG-ACC\\
    \glt{}`No, I didn't kill them.'
    \glt{}lit. `Not-me killed them.'
\end{exe}

\begin{exe}
    \ex{} \gll{}Li-li-li-li-thaka-pah kisu-ta pothu-tay,\\
    NEG-NEG-NEG-NEG-1SG-NOM kill-PST 3SG-ACC\\
    \glt{}`I definitely did not kill them.'
\end{exe}

\subsection{Relative clauses}
Thiguka uses the relative marker `si', which begins a clause used as the relative clause.
It goes after the noun phrase that is being relativized.

\begin{exe}
    \ex{} \gll{}Kasa     pekulugusag-tay, si  thaka-pah ri-kala, gulaya      ri-gulid.\\
                PROX.DET moderator-ACC    REL 1SG-NOM   INT-know COP.3SG.PRS INT-good \\
    \glt{}``This moderator, who I know, is good.''
\end{exe}

\begin{exe}
    \ex{} \gll{}Asila    pekulugusag-pah-elah pa-ithesatha-ta kasa pelu-tay-sa       si  li  ga\~{}gu-ri-palad gasari\\
                DIST.DET moderator-NOM-PL     PFV-ban-PST     PROX.DET person-ACC-PL REL NEG \agradj{}-INT-bad action\\
    \glt{}``Those moderators banned someone who did nothing wrong.''
\end{exe}

\subsection{Adverbial clauses}
Thiguka also uses the relative marker `si' for adverbial clauses. `si' is prefixed to
multiple words to further communicate the type of adverbial clause.

Temporal clauses:
\begin{exe}
    \ex{} \gll{}Thaka-pah gulaya      li-sifata ifil thaka-pafay laysa-pafay ta\~{}gu-ri-gulid  tayle\\
                1SG-NOM   COP.3SG.PRS NEG-alive in   1SG-GEN     mother-GEN  \agradj{}-INT-good time\\
          \glt{}``I was not alive during my mother's golden age.''
\end{exe}

Causal clauses:
\begin{exe}
    \ex{} \gll{}Kare-pah, si-dasili  li-kala  kolusute, gulaya      li-gu-asaga   pafay situ-pafay   lahpeled-pafay saleyfata\\
    Karen-NOM REL-reason NEG-know computer, COP.3SG.PRS NEG-ADJ-allow have  1PL.INCL-GEN company-GEN    occupation\\
    \glt{}``Karen, because she does not know computers, is not allowed to have a job at our company.''
\end{exe}
\subsection{Quantity}
To communicate quantity in Thiguka, write a quantifier before the start of a noun phrase.

\begin{exe}
    \ex{} \gll{}elah gahlirithi\\
                many violin\\
          \glt{}``many violins''
\end{exe}
In this gloss, both \textit{gahlirithi} and \textit{gahlirithielah} can be used.
The \textit{-elah} is already redundant due to the quantifier being placed in front of it.

In general, plurality suffixes are optional, even without a quantifier.

\begin{exe}
    \ex{} \gll{}isa  gahlirithi-elah\\
                many violin-SG\\
          \glt{}``*one violins''
\end{exe}
Of course, mismatching plurality suffixes are still wrong.

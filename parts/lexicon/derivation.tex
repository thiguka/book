\section{Derivation}
Thiguka employs multiple strategies to derive and coin new words.

\subsection{Derivational morphology}
Thiguka has a variety of prefixes and suffixes that one can use 

A few select examples of Thiguka derivational prefixes are:

\begin{enumerate}
    \item fah-, agent prefix akin to English -er
    \item ri-, intensifier
    \item li-, negation
    \item filay-, ``in the presence of a cat''
\end{enumerate}

\ex
\begingl
\gla  gesa-filay-katarila//
\glb  act.of.NMLZ-with.cat-space.travel.V//
\glft `act of traveling through outer space with a cat'//
\endgl
\xe

\ex
\begingl
\gla  fah-katarila//
\glb  AGT-space.travel//
\glft `space traveler'//
\endgl
\xe

\ex
\begingl
\gla  fah-filay-katarila//
\glb  AGT-with.cat-space.travel//
\glft `one who travels through space with a pet cat'//
\endgl
\xe

\ex
\begingl
\gla  fah-filay-li-katarila//
\glb  AGT-with.cat-NEG-space.travel//
\glft `one who does not travel through space with a pet cat'//
\endgl
\xe

\subsection{Long phrases}
The shortening of long strings of adjectives is a common word derivation method in Thiguka, with long word compounds being shortened.
Usually, a syllable is taken from the front of each modifying word and then attached to the final few syllables of the modified word.

\ex
\begingl
\gla   thi\~{}gu-asalisa alu thi-gu-luti thil//
\glb   AGR\~{}ADJ-sun and AGR-ADJ-moon thing//
\glft  `asalilutil'; `solar eclipse'//
\endgl
\xe

Often, longer phrases are condensed.

\ex
\begingl
\gla   pelu-pah-sa kufa-sa pelu-tay-elah kolaga-sa tha\~{}gu-ithere tharasag-tay-elah//
\glb   person-NOM-SG compel-PRS person-ACC-PL comply-PRS \agradj{}-internet rule-ACC-PL//
\glft  `pekulugusag', `internet moderator' (lit. `person who makes people comply with internet rules')//
\endgl
\xe

Thiguka also has no explicit rule against shortening and combining already-compounded noun phrases. These words are treated like roots in Thiguka and can be derived like any other.

\ex
\begingl
\gla   pe\~{}gu-palad pekulugusag//
\glb   \agradj{}-bad internet.moderator//
\glft  `bad moderator', `annoying moderator'//
\endgl
\xe

\ex
\begingl
\gla   fah-lahkela-pah pekulugusag-tay-elah//
\glb   AGT-make-NOM internet.moderator-ACC-PL//
\glft `internet moderator trainer'; lit. `maker of internet moderators'//
\endgl
\xe

For denoting administrators, who are generally ranked higher than moderators in most internet communities:

\ex
\begingl
\gla ri-pekulugusag//
\glb INT-internet.moderator//
\glft `admin'; lit. `super internet moderator'//
\endgl
\xe


\newpage
\section{Grammar}
Thiguka has an agglutinative grammar, a nominative-accusative case marking system, subject-verb-object word order, a tense-aspect system, a single copula with 37 irregular forms that inflect for time, person, and number, and a 38th irregular form that acts as a general-purpose copula.

\subsection{Plurality}
Plurality marking is optional.
If the speaker does not know the quantity of a noun, they may simply omit it.

\begin{table}[h!]
    \centering
    \caption{Plurality}
    \begin{tabularx}{8cm}{|X|X|}
        \hline
        \textbf{Plurality} & \textbf{Suffix} \\
        \hline
        None (zero) & -pa \\
        Singular & -sa \\
        Dual & -gula \\
        Plural & -elah \\
        \hline
    \end{tabularx}
\end{table}

\newpage{}

\subsection{Case}
Thiguka features six cases, with their own specific uses.
Case marking is completely optional.
A completely uninflected word is that word's dictionary form.

\begin{table}[h!]
    \centering
    \caption{Cases}
    \begin{tabularx}{8cm}{|X|X|}
        \hline
        \textbf{Case} & \textbf{Suffix} \\
        \hline
        Nominative & -pah \\
        Accusative & -tay \\
        Dative & -lay \\
        Genitive & -pafay \\
        Locative & -kala \\
        Instrumental & -kithi \\ 
        \hline
    \end{tabularx}
\end{table}

\subsubsection{Nominative}
Nominative case is used in Thiguka to mark subjects.

\subsubsection{Accusative}
Accusative case is used in Thiguka to mark objects.

\subsubsection{Dative}
Dative case is used in Thiguka, in the below examples.

\subsubsection{Genitive}
Genitive case has only been used by Lemuria in his Thiguka writings, to indicate posession.

\begin{exe}
\ex{} \gll{}Kasa-pah     gulaya     Gahlija-pafay gahlirithi.\\
            PROX.DET-NOM COP.3SG.PRS Gallia-GEN   violin\\
    \glt{}``This is Gallia's violin.''
\end{exe}

\subsubsection{Locative}
Locative case is used for a location or a time something happened.
It can be combined with the preposition \textit{ifil}.

\begin{exe}
\ex{} \gll{}Pothu-pah pa-teligara-ta        ifil Los Algeles-kala ifil literidilay-kala\\
            3SG-NOM   PFV-twitch.stream-PST in   Los Angeles-LOC  in   yesterday-LOC\\
    \glt{}``She streamed on Twitch in Los Angeles yesterday.'' (lit. ``She streamed on Twitch in Los Angeles, in yesterday'')
\end{exe}

\subsubsection{Instrumental}
Instrumental case is used to denote, in general, the methods through which the subject did something.

\begin{exe}
\ex{} \gll{}Pothu-pah pa-sapakalakesi-ta ka\~{}gu-rifari  kalakesi Lipset-kithi ifil Los Algeles-kala, Kaliforlija-kala.\\
            3SG-NOM   PFV-graduate-PST   \agradj{}-music  school   Lipsett-INS  in   Los Angeles-LOC   California-LOC.\\
    \glt{}``She graduated from the music school under Lipsett in Los Angeles, California.''
\end{exe}
In this example, perhaps Lipsett had been ``her'' primary music teacher the whole time she was at the school.

\newpage{}


\subsection{Adjectives}
Adjectives in Thiguka are placed before words. To attach an adjective to a word, the first syllable of the modified word, followed by \emph{-gu-} should be attached to the adjective.

\begin{exe}
    \ex{} \gll{}ri\~{}{}gu-gulid rifari\\
    \agradj{}-good music\\
    \glt{}`good music'
\end{exe}

Adjectives can be used as nouns too:

\begin{exe}
    \ex{} \gll{}Pothu-pafay gu-gulid-pah kufa pelu-tay-elah falusaya-sa pothu-tay\\
    3SG-GEN ADJ-good-NOM allow person-ACC-PL admire-PRS 3SG-ACC\\
    \glt{}`Their greatness made them likeable.' (Their goodness made people like them.)
\end{exe}

\subsection{Determiners}
Thiguka has proximal, medial, and distal determiners. Determiners are placed before root words. 

\begin{exe}
    \ex{} \gll{}ka\~{}{}gu-parthas \emph{kasa} kaela\\
    \agradj{}-pink.purple PROX.DET tree\\
    \glt{}`\emph{this} sakura tree'
\end{exe}

\begin{exe}
    \ex{} \gll{}ka\~{}{}gu-parthas \emph{saka} kaela\\
    \agradj{}-pink.purple MED.DET tree\\
    \glt{}`\emph{that} sakura tree \emph{near you}'
\end{exe}

\begin{exe}
    \ex{} \gll{}ka\~{}{}gu-parthas \emph{asila} kaela\\
    \agradj{}-pink.purple DIS.DET tree\\
    \glt{}`\emph{that} sakura tree'
\end{exe}

\subsection{Tense-aspect}
Thiguka has a tense-aspect system.

Tense and aspect are optional.
If tense and aspect are omitted, the verb becomes infinitive; or the time
it took place becomes ambiguous, granting it time-independence.

\begin{exe}
    \ex{} \gll{}katarila\\
    outer.space.travel\\
    \glt{}`to travel through outer space'
\end{exe}

\begin{exe}
    \ex{} \gll{}ras-katarila-ta\\
    ACCIDENT-outer.space.travel-PRS\\
    \glt{}`to have accidentally traveled through outer space'
\end{exe}

\begin{exe}
    \ex{} \gll{}ge-katarila-sa\\
    PROG-outer.space.travel-PRS\\
    \glt{}`to be currently traveling through outer space'
\end{exe}

\begin{table}[h!]
    \centering
    \caption{Tense}
    \begin{tabularx}{8cm}{|X|X|}
        \hline
        \textbf{Tense} & \textbf{Suffix} \\
        \hline
        Past & -ta \\
        Recent past & -tu \\
        Present & -sa \\
        Future & -tha \\
        \hline
    \end{tabularx}
\end{table}

\begin{table}[h!]
    \centering
    \caption{Aspect}
    \begin{tabularx}{8cm}{|X|X|}
        \hline
        \textbf{Aspect} & \textbf{Prefix} \\
        \hline
        Perfective & pa- \\
        Perfect & fe- \\
        Progressive & ge- \\
        Imperfective & sa- \\
        Contemplative & kas- \\
        Accidental & ras- \\
        Intentional & ka- \\
        \hline
    \end{tabularx}
\end{table}

\subsection{Pronouns}
Thiguka completely lacks gender distinction in pronouns, and has a clusivity distinction in first person pronouns.

\begin{table}[h!]
    \centering
    \caption{Pronouns}
    \begin{tabularx}{15cm}{|X|X|X|X|}
        \hline
        person/number & \textbf{First} & \textbf{Second} & \textbf{Third} \\
        \hline
        singular & thaka & kake & pothu \\
        plural   & & katake & rafu \\
        inclusive plural & situ & & \\
        exclusive plural & pata & & \\
        \hline
    \end{tabularx}
\end{table}

Thiguka has no T-V distinction in pronouns. Politeness may be achieved by attaching positive adjectives to second-person pronouns, but this is discouraged and goes against Lemuria's intentions.

\subsection{Template}
Thiguka positions morphemes in a specific order, based on whether it is a noun or verb.

\subsubsection{Noun}
For nouns, the order is as follows: Derivational prefixes, Negation or intensifiers, Stem, Case, Adverb suffixes, Plurality.

\subsubsection{Verb}
For verbs, the order is as follows: Derivational prefixes, Negation or intensifiers, Aspect, Stem, Tense, Adverb suffixes

\subsection{Derivation}
Thiguka has multiple methods of taking in new words into its lexicon; such as compounding and the addition of derivational prefixes.

A few select examples of Thiguka derivational prefixes are:

\begin{enumerate}
    \item fah-, agent prefix akin to English -er
    \item ri-, intensifier
    \item li-, negation
    \item filay-, ``in the presence of a cat''
\end{enumerate}

\begin{exe}
    \ex{} \gll{}fah-katarila\\
    AGT-space.travel\\
    \glt{}`space traveler'
\end{exe}

\begin{exe}
    \ex{} \gll{}fah-filay-katarila\\
    AGT-with.cat-space.travel\\
    \glt{}`one who travels through space with a pet cat'
\end{exe}

\begin{exe}
    \ex{} \gll{}fah-filay-li-katarila\\
    AGT-with.cat-NEG-space.travel\\
    \glt{}`one who does not travel through space with a pet cat'
\end{exe}

\subsection{Copula}
Thiguka has a single copula with 38 irregular forms.

\begin{table}[h!]
    \centering
    \caption{Copula forms}
    \begin{tabularx}{15cm}{|X|X|X|X|X|}
        \hline
        Person & \textbf{Past} & \textbf{Recent Past} & \textbf{Present} & \textbf{Future} \\
        \hline
        1SG & lasata & lasatu & lasa & lasatha \\
        1DU & lagulata & lagulatu & lagulasa & lagulatha \\
        1PL & lelahtaw & lelahtaw & lelahsa & lelahtha \\
        2SG & kaketa & katu & kasa & katha \\
        2DU & kagulata & kagulatu & kagulasa & kagulatha \\
        2PL & kelahta & kelahtu & kelahsa & kelahtha \\
        3SG & gulata & gulatu & gulaya & gulatha \\
        3DU & poguta & pogutu & pogusa & pogutha \\
        3PL & lotares & lotures & lores & lotheres \\
        \hline
    \end{tabularx}
\end{table}

% ! general-purpose
% | colspan=4 | las



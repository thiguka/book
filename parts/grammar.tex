
\chapter{Grammar}
Thiguka has an agglutinative grammar, a nominative-accusative case marking system, subject-verb-object word order, a tense-aspect system, a single copula with 37 irregular forms that inflect for time, person, and number, and a 38th irregular form that acts as a general-purpose copula.

\section{Plurality}
Plurality marking is optional.
If the speaker does not know the quantity of a noun, they may simply omit it.

\begin{table}[h!]
    \centering
    \caption{Plurality}
    \begin{tabularx}{8cm}{|X|X|}
        \hline
        \textbf{Plurality} & \textbf{Suffix} \\
        \hline
        None (zero) & -pa \\
        Singular & -sa \\
        Dual & -gula \\
        Plural & -elah \\
        \hline
    \end{tabularx}
\end{table}

\newpage{}

\subsection{Case}
Thiguka features six cases, with their own specific uses.
Case marking is completely optional.
A completely uninflected word is that word's dictionary form.

\begin{table}[h!]
    \centering
    \caption{Cases}
    \begin{tabularx}{8cm}{|X|X|}
        \hline
        \textbf{Case} & \textbf{Suffix} \\
        \hline
        Nominative & -pah \\
        Accusative & -tay \\
        Dative & -lay \\
        Genitive & -pafay \\
        Locative & -kala \\
        Instrumental & -kithi \\ 
        \hline
    \end{tabularx}
\end{table}

\subsubsection{Nominative}
Nominative case is used in Thiguka to mark subjects.

\subsubsection{Accusative}
Accusative case is used in Thiguka to mark objects.

\subsubsection{Dative}
Dative case is used in Thiguka, in the below examples.

\subsubsection{Genitive}
Genitive case has only been used by Lemuria in his Thiguka writings, to indicate posession.

\begin{exe}
\ex{} \gll{}Kasa-pah     gulaya     Gahlija-pafay gahlirithi.\\
            PROX.DET-NOM COP.3SG.PRS Gallia-GEN   violin\\
    \glt{}``This is Gallia's violin.''
\end{exe}

\subsubsection{Locative}
Locative case is used for a location or a time something happened.
It can be combined with the preposition \textit{ifil}.

\begin{exe}
\ex{} \gll{}Pothu-pah pa-teligara-ta        ifil Los Algeles-kala ifil literidilay-kala\\
            3SG-NOM   PFV-twitch.stream-PST in   Los Angeles-LOC  in   yesterday-LOC\\
    \glt{}``She streamed on Twitch in Los Angeles yesterday.'' (lit. ``She streamed on Twitch in Los Angeles, in yesterday'')
\end{exe}

\subsubsection{Instrumental}
Instrumental case is used to denote, in general, the methods through which the subject did something.

\begin{exe}
\ex{} \gll{}Pothu-pah pa-sapakalakesi-ta ka\~{}gu-rifari  kalakesi Lipset-kithi ifil Los Algeles-kala, Kaliforlija-kala.\\
            3SG-NOM   PFV-graduate-PST   \agradj{}-music  school   Lipsett-INS  in   Los Angeles-LOC   California-LOC.\\
    \glt{}``She graduated from the music school under Lipsett in Los Angeles, California.''
\end{exe}
In this example, perhaps Lipsett had been ``her'' primary music teacher the whole time she was at the school.

\newpage{}



\subsection{Adjectives}
Adjectives in Thiguka are placed before words.
To attach an adjective to a word, the first syllable of the modified word, followed by \emph{-gu-} should be prefixed to the adjective.

There is no enforced adjective order in Thiguka. Speakers may position adjectives as necessary to highlight particularly important qualities.

\begin{exe}
    \ex{} \gll{}ri\~{}{}gu-gulid rifari\\
    \agradj{}-good music\\
    \glt{}`good music'
\end{exe}

Adjectives can be used as nouns too:

\begin{exe}
    \ex{} \gll{}Pothu-pafay gu-gulid-pah kufa pelu-tay-elah falusaya-sa pothu-tay\\
    3SG-GEN ADJ-good-NOM allow person-ACC-PL admire-PRS 3SG-ACC\\
    \glt{}`Their greatness made them likeable.' (Their goodness made people like them.)
\end{exe}

Speakers may also choose to include additional syllables if multiple words begin with the same syllable.
\begin{exe}
    \ex{} \gll{}Fulafata-sa, pafay-sa thi\~{}gu-sapara thil-elah, alu ara\~{}gu-li-saleyfata arafat-elah alu fulahthi lores \textbf{lisu}\~{}gu-furos li-lahkela li\~{}gu-pera \textbf{lisu}re.\\
    hike-PRS, have-PRS AGR\~{}ADJ-type thing-PL, and AGR\~{}ADJ-NEG-occupation art-PL and fun.make.thing-PL COP.3PL.PRS AGR\~{}ADJ-fun NEG-make AGR\~{}ADJ-money fun.activity.\\
    \glt{}`Hiking, having themed objects, and non-professional art are fun not-money-making hobbies.'
\end{exe}    

In this example, \textit{furos} is modifying \textit{lisure}, but to use \textit{*ligufuros} would have caused ambiguity due to the presence of \textit{lilahkela} and \textit{ligupera} in the sentence.

\subsubsection{Modifying multiple words}
To use a single adjective to modify multiple words, two approaches are possible.

Either syllables of all the words modified can be included, or the adjective can be duplicated for each word that needs to be modified.

If including multiple syllables of the modified words, the order of the syllables must match the order in the sentence.

(Adapted from one of Zethar's translations)
\begin{exe}
\ex{} \gll{}ka\~{}fu\~{}gu-dalata       thaka-pah kas-gasari     alu filay-fulafata\\
            AGR\~{}AGR\~{}ADJ-two 1SG-NOM   CONTEM-action  and with.cat-hike\\
      \glt{}Secondly, I practice and hike with the local cats.
\end{exe}

In this sentence, \textit{*fukagudalata} would be incorrect as it is not presenting the adjectives in the order that they are being used in the sentence.

(Lemuria's version)
\begin{exe}
\ex{} \gll{}ka\~{}gu-dalata thaka-pah kas-gasari alu fu\~{}gu-dalata filay-fulafata\\
            AGR\~{}ADJ-two  1SG-NOM   CONTEM-act and AGR\~{}ADJ-two  with.cat-hike\\
      \glt{}Secondly, I practice, and secondly, I hike with cats.
\end{exe}

However, the approach of using the same adjective twice causes a semantic change in meaning.
It may simply be another thing that gets ``lost in translation''.

\begin{quote}
    It’s not clear what one needs to do for adjectives modifying more than one thing.

    --- Zethar, Ambagame3
\end{quote}

\section{Determiners}
Thiguka has proximal, medial, and distal determiners. Determiners are placed before root words. 

\ex
\begingl
    \gla ka\~{}{}gu-parthas \emph{kasa} kaela//  
    \glb \agradj{}-pink.purple PROX.DET tree//  
    \glft `\emph{this} sakura tree'//  
\endgl
\xe

\ex
\begingl
    \gla ka\~{}{}gu-parthas \emph{saka} kaela//  
    \glb \agradj{}-pink.purple MED.DET tree//  
    \glft `\emph{that} sakura tree \emph{near you}//  
\endgl
\xe

\ex
\begingl
    \gla ka\~{}{}gu-parthas \emph{asila} kaela//  
    \glb \agradj{}-pink.purple DIS.DET tree//  
    \glft `\emph{that} sakura tree'//  
\endgl
\xe

\section{Tense-aspect}
Thiguka has a tense-aspect system.

Tense and aspect are optional.
If tense and aspect are omitted, the verb becomes infinitive; or the time
it took place becomes ambiguous, granting it time-independence.

\ex
\begingl
    \gla katarila//
    \glb outer.space.travel//
    \glft `to travel through outer space'//
\endgl
\xe

\ex
\begingl
    \gla ras-katarila-ta//
    \glb ACCIDENT-outer.space.travel-PRS//
    \glft `to have accidentally traveled through outer space'//
\endgl
\xe

\ex
\begingl
    \gla ge-katarila-sa//
    \glb PROG-outer.space.travel-PRS//
    \glft `to be currently traveling through outer space'//
\endgl
\xe

\begin{table}[h!]
    \centering
    \caption{Tense}
    \begin{tabularx}{8cm}{|X|X|}
        \hline
        \textbf{Tense} & \textbf{Suffix} \\
        \hline
        Past & -ta \\
        Recent past & -tu \\
        Present & -sa \\
        Future & -tha \\
        \hline
    \end{tabularx}
\end{table}

\begin{table}[h!]
    \centering
    \caption{Aspect}
    \begin{tabularx}{8cm}{|X|X|}
        \hline
        \textbf{Aspect} & \textbf{Prefix} \\
        \hline
        Perfective & pa- \\
        Perfect & fe- \\
        Progressive & ge- \\
        Imperfective & sa- \\
        Contemplative & kas- \\
        Accidental & ras- \\
        Intentional & ka- \\
        \hline
    \end{tabularx}
\end{table}

\section{Pronouns}
Thiguka completely lacks gender distinction in pronouns, and has a clusivity distinction in first person pronouns.

\begin{table}[h!]
    \centering
    \caption{Pronouns}
    \begin{tabularx}{15cm}{|X|X|X|X|}
        \hline
        person/number & \textbf{First} & \textbf{Second} & \textbf{Third} \\
        \hline
        singular & thaka & kake & pothu \\
        plural   & & katake & rafu \\
        inclusive plural & situ & & \\
        exclusive plural & pata & & \\
        \hline
    \end{tabularx}
\end{table}

Thiguka has no T-V distinction in pronouns. Politeness may be achieved by attaching positive adjectives to second-person pronouns, but this is discouraged and goes against Lemuria's intentions.



\section{Verbs}
\subsection{Morpheme order}
For verbs, the order is as follows: Derivational prefixes, Negation or intensifiers, Aspect, Stem, Tense, Adverb suffixes

\subsection{Transitivity}
Thiguka's verbs will vary in transitivity. Early versions of Thiguka's Anki lexicon may not have sufficient transitivity tagging.

\subsection{Stative verbs}
Stative verbs can be expressed through a progressive verb.

A copula here like \textit{gulaya} is optional.

\ex
\begingl
    \gla Pothu-pah (gulaya)      ge-parakala-sa gahlirithi-tay.//
    \glb 3SG-NOM   (COP.3SG.PRS) PROG-learn-PRS violin-ACC//
    \glft ``She is learning the violin.''//
\endgl
\xe

\ex
\begingl
    \gla Pothu-pah ge-parakala-tha gahlirithi-tay.//
    \glb 3SG-NOM   PROG-learn-FUT violin-ACC//
    \glft ``She will soon be learning the violin.''//
\endgl
\xe

Once she's learned the violin, one can say:

\ex
\begingl
    \gla Pothu-pah ge-kala-sa    gahlirithi-tay.//
    \glb 3SG-NOM   PROG-know-PRS violin-ACC//
    \glft ``She knows the violin.''//
\endgl
\xe

\ex
\begingl
    \gla Pothu-pah ge-pafay-sa    ka\~{}gu-gahlirithi kala-tay.//
    \glb 3SG-NOM   PROG-know-PRS \agradj{}violin knowledge-ACC//
    \glft ``She has knowledge of violins.''//
\endgl
\xe


\section{Examples}
Some examples of verbs will follow.

parakala (``to study; to learn; to receive knowledge'')

\ex
\begingl
    \gla Pothu-pah ge-parakala-sa gahlirithi-tay Olifia-kithi.//
    \glb 3SG-NOM   PROG-learn-PRS violin-ACC Olivia-INS//
    \glft ``She is learning the violin with Olivia.'' (transitive, +instrumental)//
\endgl
\xe

\ex
\begingl
    \gla Pothu-pah ge-parakala-sa gahlirithi-tay Beti-kithi ifil ka\~{}gu-rifari kalakesi-kala ifil Sikago-kala.//
    \glb 3SG-NOM   PROG-learn-PRS violin-ACC Betty-INS in \agradj{}-music school-LOC in Chicago-LOC//
    \glft ``She is learning the violin with Betty at a school in Chicago.'' (transitive, +locative, +instrumental)//
\endgl
\xe

\ex
\begingl
    \gla Pothu-pah ge-parakala-sa.//
    \glb 3SG-NOM   PROG-learn-PRS//
    \glft ``She is learning.'' (intransitive)//
\endgl
\xe

\section{Adverbs}
In Thiguka, adverbs are either placed before the verb they modify or as clitics to the end of the verb.

These adverbs may be marked with the marker \textit{-se}, but can be omitted if there is sufficient context.

\ex
\begingl
\gla   Rafu-pah pa-leyle-ta=thasi rafu-pafay   balisa-lay-sa   Katrila-tay.//
\glb   3PL-NOM  PFV-name-PST=also 3PL-GEN.POSS daughter-DAT-SG Katrina-ACC//
\glft  ``They also named their daughter Katrina.''//
\endgl
\xe


However, \textit{-se} is mandatory when adverbializing an adjective:

\ex
\begingl
\glpreamble Usage of \textit{free} as an adverb:// 
\gla   Thaka-pah pa-paratas-ta=   @ li-pera-      @ se      kasa kolusute-tay-sa//
\glb   1SG-NOM   PFV-receive-PST=   NEG-money.ADV   -ADVZ   this computer-ACC-SG//
\glft  I received this computer for free.//
\endgl
\xe


\section{Template}
Thiguka positions morphemes in a specific order, based on whether it is a noun or verb.

\subsection{Noun}
For nouns, the order is as follows: Derivational prefixes, Negation or intensifiers, Stem, Case, Adverb suffixes, Plurality.

\subsection{Verb}
For verbs, the order is as follows: Derivational prefixes, Negation or intensifiers, Aspect, Stem, Tense, Adverb suffixes

\section{Derivation}
Thiguka has multiple methods of taking in new words into its lexicon; such as compounding and the addition of derivational prefixes.

A few select examples of Thiguka derivational prefixes are:

\begin{enumerate}
    \item fah-, agent prefix akin to English -er
    \item ri-, intensifier
    \item li-, negation
    \item filay-, ``in the presence of a cat''
\end{enumerate}

\ex
\begingl
    \gla  fah-katarila//
    \glb  AGT-space.travel//
    \glft `space traveler'//
\endgl
\xe

\ex
\begingl
    \gla  fah-filay-katarila//
    \glb  AGT-with.cat-space.travel//
    \glft `one who travels through space with a pet cat'//
\endgl
\xe

\ex
\begingl
    \gla  fah-filay-li-katarila//
    \glb  AGT-with.cat-NEG-space.travel//
    \glft `one who does not travel through space with a pet cat'//
\endgl
\xe

\section{Copula}
Thiguka has a single copula with 38 irregular forms.

\begin{table}[h!]
    \centering
    \caption{Copula forms}
    \begin{tabularx}{15cm}{|X|X|X|X|X|}
        \hline
        Person & \textbf{Past} & \textbf{Recent Past} & \textbf{Present} & \textbf{Future} \\
        \hline
        1SG & lasata & lasatu & lasa & lasatha \\
        1DU & lagulata & lagulatu & lagulasa & lagulatha \\
        1PL & lelahtaw & lelahtaw & lelahsa & lelahtha \\
        2SG & kaketa & katu & kasa & katha \\
        2DU & kagulata & kagulatu & kagulasa & kagulatha \\
        2PL & kelahta & kelahtu & kelahsa & kelahtha \\
        3SG & gulata & gulatu & gulaya & gulatha \\
        3DU & poguta & pogutu & pogusa & pogutha \\
        3PL & lotares & lotures & lores & lotheres \\
        \hline
    \end{tabularx}
\end{table}

% ! general-purpose
% | colspan=4 | las



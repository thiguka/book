
\newpage
\section{Grammar}
Thiguka has an agglutinative grammar, a nominative-accusative case marking system, subject-verb-object word order, a tense-aspect system, a single copula with 37 irregular forms that inflect for time, person, and number, and a 38th irregular form that acts as a general-purpose copula.

\subsection{Plurality}
Plurality marking is optional.

\begin{table}[h!]
    \centering
    \begin{tabularx}{8cm}{|X|X|}
        \hline
        \textbf{Plurality} & \textbf{Suffix} \\
        \hline
        None (zero) & -pa \\
        Singular & -sa \\
        Dual & -gula \\
        Plural & -elah \\
        \hline
    \end{tabularx}
    \caption{Plurality}
\end{table}

\subsection{Case}
\begin{table}[h!]
    \centering
    \begin{tabularx}{8cm}{|X|X|}
        \hline
        \textbf{Case} & \textbf{Suffix} \\
        \hline
        Nominative & -pah \\
        Accusative & -tay \\
        Dative & -lay \\
        Genitive & -pafay \\
        Locative & -kala \\
        Instrumental & -kithi \\ 
        \hline
    \end{tabularx}
    \caption{Cases}
\end{table}

\subsection{Adjectives}
Adjectives in Thiguka are placed before words. To attach an adjective to a word, the first syllable of the modified word, followed by \emph{-gu-} should be attached to the adjective.

\begin{exe}
    \ex{} \gll{}ri\textapprox{}gu-gulid rifari\\
    \agradj{}-good music\\
    \glt{}`good music'
\end{exe}

\subsection{Determiners}
Thiguka has proximal, medial, and distal determiners. Determiners are placed before root words. 

\begin{exe}
    \ex{} \gll{}ka\textapprox{}gu-parthas \emph{kasa} kaela\\
    \agradj{}-pink.purple PROX.DET tree\\
    \glt{}`\emph{this} sakura tree'
\end{exe}

\begin{exe}
    \ex{} \gll{}ka\textapprox{}gu-parthas \emph{saka} kaela\\
    \agradj{}-pink.purple MED.DET tree\\
    \glt{}`\emph{that} sakura tree \emph{near you}'
\end{exe}

\begin{exe}
    \ex{} \gll{}ka\textapprox{}gu-parthas \emph{asila} kaela\\
    \agradj{}-pink.purple DIS.DET tree\\
    \glt{}`\emph{that} sakura tree'
\end{exe}

\subsection{Tense-aspect}
Thiguka has a tense-aspect system.

\begin{table}[h!]
    \centering
    \begin{tabularx}{8cm}{|X|X|}
        \hline
        \textbf{Tense} & \textbf{Suffix} \\
        \hline
        Past & -ta \\
        Recent past & -tu \\
        Present & -sa \\
        Future & -tha \\
        \hline
    \end{tabularx}
    \caption{Tense}
\end{table}

\begin{table}[h!]
    \centering
    \begin{tabularx}{8cm}{|X|X|}
        \hline
        \textbf{Aspect} & \textbf{Prefix} \\
        \hline
        Perfective & pa- \\
        Perfect & fe- \\
        Progressive & ge- \\
        Imperfective & sa- \\
        Contemplative & kas- \\
        Accidental & ras- \\
        Intentional & ka- \\
        \hline
    \end{tabularx}
    \caption{Aspect}
\end{table}

\subsection{Pronouns}
Thiguka completely lacks gender distinction in pronouns, and has a clusivity distinction in first person pronouns.

\begin{table}[h!]
    \centering
    \begin{tabularx}{15cm}{|X|X|X|X|}
        \hline
        person/number & \textbf{First} & \textbf{Second} & \textbf{Third} \\
        \hline
        singular & thaka & kake & pothu \\
        plural   & & katake & rafu \\
        inclusive plural & situ & & \\
        exclusive plural & pata & & \\
        \hline
    \end{tabularx}
    \caption{Pronouns}
\end{table}

Thiguka has no T-V distinction in pronouns. Politeness may be achieved by attaching positive adjectives to second-person pronouns, but this is discouraged and goes against Lemuria's intentions.

\subsection{Template}
Thiguka positions morphemes in a specific order, based on whether it is a noun or verb.

\subsubsection{Noun}


\subsubsection{Verb}


\subsection{Derivation}
Thiguka has multiple methods of taking in new words into its lexicon; such as compounding and the addition of derivational prefixes.

A few select examples of Thiguka derivational prefixes are:

\begin{enumerate}
    \item fah-, agent prefix akin to English -er
    \item ri-, intensifier
    \item li-, negation
    \item filay-, ``in the presence of a cat''
\end{enumerate}

\begin{exe}
    \ex{} \gll{}fah-katarila\\
    AGT-space.travel\\
    \glt{}`space traveler'
\end{exe}

\begin{exe}
    \ex{} \gll{}fah-filay-katarila\\
    AGT-with.cat-space.travel\\
    \glt{}`one who travels through space with a pet cat'
\end{exe}

\begin{exe}
    \ex{} \gll{}fah-filay-li-katarila\\
    AGT-with.cat-NEG-space.travel\\
    \glt{}`one who does not travel through space with a pet cat'
\end{exe}


\newpage

\section{Phonology}

Perhaps one of the rarest characteristics of Thiguka phonology is its complete lack of phonemic nasals.

Thiguka has 11 consonants, 6 vowels, and 6 diphthongs.


\begin{table}[h!]
    \centering
    \begin{tabularx}{15cm}{|X|X|X|X|X|X|X|}
        \hline
                                     & \textbf{Bilabial} & \textbf{Labiodental} & \textbf{Dental} & \textbf{Alveolar} & \textbf{Velar} \\
        \hline
        \textbf{Plosive}             & p, b              &                      &                 & t, d              & k, g           \\
        \hline
        \textbf{Fricative}           &                   & f                    & \textipa{T}     & s                 &                \\
        \hline
        \textbf{Rhotic}              &                   &                      &                 & r                 &                \\
        \hline
        \textbf{Lateral}             &                   &                      &                 & l                 &                \\
        \hline
    \end{tabularx}
    \caption{Consonant Phonemes}
\end{table}

\begin{table}[h!]
    \centering
    \begin{tabularx}{8cm}{|X|X|X|X|X|}
        \hline
                           & \textbf{Front} & \textbf{Back} \\
        \hline
        \textbf{Close}     & i              & u             \\
        \hline
        \textbf{Close-Mid} & e              & o             \\
        \hline
        \textbf{Open}      & a              & \textipa{A}   \\
        \hline
    \end{tabularx}
    \caption{Vowel Phonemes}
\end{table}

\begin{table}[h!]
    \centering
    \begin{tabularx}{\textwidth}{|X|X|X|X|X|X|X|}
        \hline
        \textbf{Diphthongs} & ai & ei & ui & au & \textipa{Ai} & ia \\
        \hline
    \end{tabularx}
    \caption{Diphthongs}
\end{table}

The [r] consonant is realized by Lemuria as the English approximant, however one may also pronounce it as a flap or trill if desired.

% Thiguka inserts a glottal stop \textipa{[ʔ]}

[ia] is very rare in a priori Thiguka words, being exclusively used for loanwords such as Enya (Elija).

\subsection{Nasals}

Thiguka completely lacks nasals. [n] is an allophone of [l], while [m] is an allophone of [b], and [ŋ] is approximated as any one of [lg g k l ɫ].
The chosen consonant differs depending on the context the velar nasal is in; for example, a speaker would not want to use [g] in for `fang' [faŋ], because then they would be uttering a racial slur.

\begin{table}[h!]
    \centering
    \caption{Examples of nasal consonant approximations in Thiguka}
    \begin{tabularx}{15cm}{|X|X|X|X|}
        \hline
        \textbf{English} & \textbf{English IPA} & \textbf{Thiguka} & \textbf{Thiguka IPA} \\
        \hline
        Lemuria    & [li.ˈmɜɹ.iə]   & Liburija  & [ˈli.bur.ia] \\
        Enya       & [ˈɛn.jə]       & Elija     & ['elia] \\
        Lena Raine & [ˈlɛnə ˈreɪn]  & Lela Reyl & [ˈlela reɪl] \\
        \hline
    \end{tabularx}
\end{table}

\subsection{Stress}
Thiguka places stress on the first syllable of the root word.
However, stress is not a phonemic feature in Thiguka, and English speakers speaking Thiguka with English stress rules will be readily understood.

\subsection{Phonotactics}
Thiguka syllable structure is (C)(C)V(C).

Words may not end in [k, r, l].
If due to loanwords, or conjugations the word ends up ending in these three forbidden consonants, either add a dummy -a suffix, or remove the offending consonant.

Words may not end in consonant clusters.

Two diphthongs separated by a single consonant are not allowed --- *aybay. 

Vowels next to each other, that are not diphthongs, are pronounced with a glottal stop inserted between them.

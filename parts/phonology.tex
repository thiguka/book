\newpage

\section{Phonology}

Perhaps one of the rarest characteristics of Thiguka phonology is its complete lack of phonemic nasals.

Thiguka has 11 consonants, 6 vowels, and 6 diphthongs.

\input{parts/phonology/table.tex}

% Thiguka inserts a glottal stop \textipa{[ʔ]}

[ia] is very rare in a priori Thiguka words, being exclusively used for loanwords such as Enya (Elija).

\subsection{Nasals}

Thiguka completely lacks nasals. [n] is an allophone of [l], while [m] is an allophone of [b], and [\textipa{N}] is approximated as any one of [\textipa{g, lg, k, l}].
The chosen consonant differs depending on the context the velar nasal is in; for example, a speaker would not want to use [g] in for `fang' [\textipa{faN}], because then they would be uttering a racial slur.

\begin{table}[h!]
    \centering
    \begin{tabularx}{15cm}{|X|X|X|X|}
        \hline
        \textbf{English} & \textbf{English IPA} & \textbf{Thiguka} & \textbf{Thiguka IPA} \\
        \hline
        Lemuria & \textipa{[li.'m3r.i@]} & Liburija  & \textipa{['li.bur.ia]} \\
        Enya & \textipa{['En.ja]} & Elija & ['elia] \\
        Lena Raine & \textipa{['lEn@ 'reIn]} & Lela Reyl & \textipa{['lela 'reIl]} \\
        \hline
    \end{tabularx}
    \caption{Examples of nasal consonant approximations in Thiguka}
\end{table}

\subsection{Stress}
Thiguka places stress on the first syllable of the root word.
However, stress is not a phonemic feature in Thiguka, and English speakers speaking Thiguka with English stress rules will be readily understood.

\subsection{Phonotactics}
Thiguka syllable structure is (C)(C)V(C).

Words may not end in [k, r, l].
If due to loanwords, or conjugations the word ends up ending in these three forbidden consonants, either add a dummy -a suffix, or remove the offending consonant.

Words may not end in consonant clusters.

Two diphthongs separated by a single consonant are not allowed --- *aybay. 

Vowels next to each other, that are not diphthongs, are pronounced with a glottal stop inserted between them.

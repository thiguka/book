
\newpage
\section{Lexicon}
Thiguka, as of May 2024, has about 350 defined words, with slight bias towards Internet and music-related terminology.

Thiguka's full lexicon is available at \url{https://github.com/thiguka/lexicon} in both tab-separated values format and an Anki deck.
A select few words from Thiguka's lexicon are as follows:

\subsection{Swadesh-Yakhontov list}
Because of the author's laziness and some holes in the lexicon, he is only able to provide a 35-word Swadesh-Yakhontov list below:

\begin{multicols}{2} 
\begin{enumerate}
    \item I --- thaka
    \item you (singular) --- kake
    \item this --- kasa
    \item who --- kalisu pelu
    \item what --- kalisu
    \item one --- isa 
    \item two --- dalata
    \item fish --- gothi
    \item dog --- aso
    \item louse --- lafauy
    \item blood --- rahsa
    \item bone --- bole
    \item egg --- logi
    \item horn --- kipi
    \item tail --- taysa
    \item ear --- eyla
    \item eye --- fathas
    \item nose --- fothes
    \item tooth --- tutho
    \item tongue --- kageta
    \item hand --- aysi
    \item know --- kala
    \item die --- dasay
    \item give --- salath
    \item sun --- asalisa
    \item moon --- luti
    \item water --- tubi
    \item salt --- salu
    \item stone --- sute
    \item wind --- lufah
    \item fire --- apay
    \item year --- fasate
    \item current year --- saro
    \item full --- hakath
    \item new --- bage
    \item name --- leyle
\end{enumerate}
\end{multicols}

\subsection{Unusual roots}
Thiguka has some specific roots that many natlangs would need an entire sentence or even paragraph to describe.
These roots are an extension of Lemuria's personality, condensing the things and concepts of significance to his life into one word.

\begin{multicols}{2} 
\begin{enumerate}
    \item ketala --- fear of being banned by a moderator on the Internet while asleep
    \item filay- --- from English feline; prefix to indicate ``with cat present''
    \item katarila --- space travel; because Lemuria asked someone (Catarina) in the Conlangs Discord Network too many questions about 3SG.GEN far-future world.
    \item orielija --- to sail away. Combination of Orinoco + Enya (whose name gets loaned as Elija due to Thiguka being a Lo Lasals Lalguage.)
\end{enumerate}
\end{multicols}

\subsection{Interesting etymologies}
Thiguka has some specific roots with interesting etymologies.

\begin{multicols}{2} 
    \begin{enumerate}
        \item lirisara --- to forget something completely obvious; -- coined when in the Conlangs Discord Network, Lemuria asked what a "saralang" was in the Hiidaden thread of "projects", failing to realize that the creator's name was Sarah.
        \item gahlirithi --- a violin; -- short form of Gahlijapafay lahrithi ("Gallia's music instrument"), from Gallia Kastner, a violinist who appeared in Lemuria's recommended one day.
        \item ukagahli --- to throw a sheep into a river; -- because Lemuria observed Gallia throw a sheep into a river while playing a video game in June 2024. This itself is also an unusual root.
    \end{enumerate}
\end{multicols}

\subsection{Twitch}
Thiguka has its fair share of Twitch-related terminology.
Lemuria knows a lot of indie musicians who sing for their viewers on the site and of course, can't help but have words just for Twitch.

\begin{multicols}{2} 
    \begin{enumerate}
        \item Telits --- Twitch
        \item Telitslisa --- n. fem. Female Twitch streamer
        \item Telitstoli --- n. masc. Male Twitch streamer
        \item Telitspelu --- n. epicene; Twitch streamer
        \item teligara --- v. to stream on Twitch
        \item telufefi --- n. a Twitch raid; when a streamer moves their viewers to another stream after ending; blend of Telits ("Twitch") + "ofefi" ("to move")
        \item teliri --- v. to fall asleep while watching a Twitch stream
        \item telires --- n. along a Twitch raid train (directionality)
    \end{enumerate}
\end{multicols}

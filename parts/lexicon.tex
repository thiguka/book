
\newpage
\section{Lexicon}
Thiguka, as of May 2024, has about 350 defined words, with slight bias towards Internet and music-related terminology.

Thiguka's full lexicon for the purposes of the Ambagame3 contest is available at \url{https://github.com/thiguka/lexicon/tree/a4fc879f8299949b4f6c7be7293349ec96e470aa} in both tab-separated values format (works with a spreadsheet program like Excel or LibreOffice, hopefully) and an Anki deck.

The URL provided links to the state of the repository at a certain point in time. Do NOT use \url{https://github.com/thiguka/lexicon}, as this is the living version and will likely be updated during the contest.

You should have received this PDF file and a TSV alongside some instructions from a CDN admin like Zethar - follow their instructions for how to proceed.

A select few words from Thiguka's lexicon are as follows:

\subsection{Swadesh-Yakhontov list}
Because of the author's laziness and some holes in the lexicon, he is only able to provide a 35-word Swadesh-Yakhontov list below:

\begin{multicols}{2} 
\begin{enumerate}
    \item I --- thaka
    \item you (singular) --- kake
    \item this --- kasa
    \item who --- kalisu pelu
    \item what --- kalisu
    \item one --- isa 
    \item two --- dalata
    \item fish --- gothi
    \item dog --- aso
    \item louse --- lafauy
    \item blood --- rahsa
    \item bone --- bole
    \item egg --- logi
    \item horn --- kipi
    \item tail --- taysa
    \item ear --- eyla
    \item eye --- fathas
    \item nose --- fothes
    \item tooth --- tutho
    \item tongue --- kageta
    \item hand --- aysi
    \item know --- kala
    \item die --- dasay
    \item give --- salath
    \item sun --- asalisa
    \item moon --- luti
    \item water --- tubi
    \item salt --- salu
    \item stone --- sute
    \item wind --- lufah
    \item fire --- apay
    \item year --- fasate
    \item current year --- saro
    \item full --- hakath
    \item new --- bage
    \item name --- leyle
\end{enumerate}
\end{multicols}

\newpage{}
\subsection{Unusual roots}
Thiguka has some specific roots that many natlangs would need an entire sentence or even paragraph to describe.
These roots are an extension of Lemuria's personality, condensing the things and concepts of significance to his life into one word.

\begin{multicols}{2} 
\begin{enumerate}
    \item ketala --- fear of being banned by a moderator on the Internet while asleep
    \item filay- --- from English feline; prefix to indicate ``with cat present''
    \item katarila --- space travel; because Lemuria asked someone (Catarina) in the Conlangs Discord Network too many questions about 3SG.GEN far-future world.
    \item orielija --- to sail away. Combination of Orinoco + Enya (whose name gets loaned as Elija due to Thiguka being a Lo Lasals Lalguage.)
\end{enumerate}
\end{multicols}

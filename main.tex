\documentclass{thigukabook}
\usepackage{bookmark}
\usepackage{multirow}
\usepackage{tabularx}
\usepackage{lmodern}
\usepackage{multicol}

\usepackage{expex}
\lingset{everygla=, belowglpreambleskip=-0.5ex, aboveglftskip=-0.5ex} % gloss formatting

\usepackage[margin=4em]{geometry}

\usepackage{array}
\usepackage{tipa}
\usepackage{textcomp}

\usepackage{fontspec}
    \setmainfont{Gentium Plus}

% Change this filename to the name of whatever you're writing about the language.

% Thiguka is in the public domain; and so is its documentation!


\title{The Thiguka Language}
\date{June 2024}
\author{Lemuria}

\begin{document}

\maketitle

\newpage

\thigukacopyright{}


\section*{About}
Thiguka is a constructed language that Lemuria has continued to develop since April 2024.
This contains a full description of Thiguka's grammar, ready for your reading pleasure.

\section*{Glossing abbreviations}
AGR~ADJ is used for adjective agreement.

\ex
\begingl
    \gla  tu\~{}{}gu-elo tubi//
    \glb  AGR\~{}ADJ-yellow water//
    \glft `urine'//
\endgl
\xe

PROX is used for the proximal determiner.

MED is used for the medial determiner.

DIS is used for the distal determiner.

\section*{Acknowledgements}
Thank you for reading the Thiguka documentation.

There are some people who have (constructively) criticized Thiguka; the criticism I want and need to make Thiguka a better and more well-documented language.

So far the only person to offer any extensive criticism is Zethar of the Conlangs Discord Network; in June 2024, I took part in a translation relay that Zethar hosted.
At the end of the contest I received about a page's worth of criticism, which I used to further improve Thiguka's documentation.

\tableofcontents

\part{Language}

\newpage

\section{Phonology}

Perhaps one of the rarest characteristics of Thiguka phonology is its complete lack of phonemic nasals.

Thiguka has 11 consonants, 6 vowels, and 6 diphthongs.


\begin{table}[h!]
    \centering
    \begin{tabularx}{15cm}{|X|X|X|X|X|X|X|}
        \hline
                                     & \textbf{Bilabial} & \textbf{Labiodental} & \textbf{Dental} & \textbf{Alveolar} & \textbf{Velar} \\
        \hline
        \textbf{Plosive}             & p, b              &                      &                 & t, d              & k, g           \\
        \hline
        \textbf{Fricative}           &                   & f                    & \textipa{T}     & s                 &                \\
        \hline
        \textbf{Rhotic}              &                   &                      &                 & r                 &                \\
        \hline
        \textbf{Lateral}             &                   &                      &                 & l                 &                \\
        \hline
    \end{tabularx}
    \caption{Consonant Phonemes}
\end{table}

\begin{table}[h!]
    \centering
    \begin{tabularx}{8cm}{|X|X|X|X|X|}
        \hline
                           & \textbf{Front} & \textbf{Back} \\
        \hline
        \textbf{Close}     & i              & u             \\
        \hline
        \textbf{Close-Mid} & e              & o             \\
        \hline
        \textbf{Open}      & a              & \textipa{A}   \\
        \hline
    \end{tabularx}
    \caption{Vowel Phonemes}
\end{table}

\begin{table}[h!]
    \centering
    \begin{tabularx}{\textwidth}{|X|X|X|X|X|X|X|}
        \hline
        \textbf{Diphthongs} & ai & ei & ui & au & \textipa{Ai} & ia \\
        \hline
    \end{tabularx}
    \caption{Diphthongs}
\end{table}

The [r] consonant is realized by Lemuria as the English approximant, however one may also pronounce it as a flap or trill if desired.

% Thiguka inserts a glottal stop \textipa{[ʔ]}

[ia] is very rare in a priori Thiguka words, being exclusively used for loanwords such as Enya (Elija).

\subsection{Nasals}

Thiguka completely lacks nasals. [n] is an allophone of [l], while [m] is an allophone of [b], and [\textipa{N}] is approximated as any one of [\textipa{g, lg, k, l}].
The chosen consonant differs depending on the context the velar nasal is in; for example, a speaker would not want to use [g] in for `fang' [\textipa{faN}], because then they would be uttering a racial slur.

\begin{table}[h!]
    \centering
    \caption{Examples of nasal consonant approximations in Thiguka}
    \begin{tabularx}{15cm}{|X|X|X|X|}
        \hline
        \textbf{English} & \textbf{English IPA} & \textbf{Thiguka} & \textbf{Thiguka IPA} \\
        \hline
        Lemuria & \textipa{[li.'m3r.i@]} & Liburija  & \textipa{['li.bur.ia]} \\
        Enya & \textipa{['En.ja]} & Elija & ['elia] \\
        Lena Raine & \textipa{['lEn@ 'reIn]} & Lela Reyl & \textipa{['lela 'reIl]} \\
        \hline
    \end{tabularx}
\end{table}

\subsection{Stress}
Thiguka places stress on the first syllable of the root word.
However, stress is not a phonemic feature in Thiguka, and English speakers speaking Thiguka with English stress rules will be readily understood.

\subsection{Phonotactics}
Thiguka syllable structure is (C)(C)V(C).

Words may not end in [k, r, l].
If due to loanwords, or conjugations the word ends up ending in these three forbidden consonants, either add a dummy -a suffix, or remove the offending consonant.

Words may not end in consonant clusters.

Two diphthongs separated by a single consonant are not allowed --- *aybay. 

Vowels next to each other, that are not diphthongs, are pronounced with a glottal stop inserted between them.


\newpage
\section{Grammar}
Thiguka has an agglutinative grammar, a nominative-accusative case marking system, subject-verb-object word order, a tense-aspect system, a single copula with 37 irregular forms that inflect for time, person, and number, and a 38th irregular form that acts as a general-purpose copula.

\subsection{Plurality}
Plurality marking is optional.

\begin{table}[h!]
    \centering
    \caption{Plurality}
    \begin{tabularx}{8cm}{|X|X|}
        \hline
        \textbf{Plurality} & \textbf{Suffix} \\
        \hline
        None (zero) & -pa \\
        Singular & -sa \\
        Dual & -gula \\
        Plural & -elah \\
        \hline
    \end{tabularx}
\end{table}

\subsection{Case}
\begin{table}[h!]
    \centering
    \caption{Cases}
    \begin{tabularx}{8cm}{|X|X|}
        \hline
        \textbf{Case} & \textbf{Suffix} \\
        \hline
        Nominative & -pah \\
        Accusative & -tay \\
        Dative & -lay \\
        Genitive & -pafay \\
        Locative & -kala \\
        Instrumental & -kithi \\ 
        \hline
    \end{tabularx}
\end{table}

\subsection{Adjectives}
Adjectives in Thiguka are placed before words. To attach an adjective to a word, the first syllable of the modified word, followed by \emph{-gu-} should be attached to the adjective.

\begin{exe}
    \ex{} \gll{}ri\textapprox{}gu-gulid rifari\\
    \agradj{}-good music\\
    \glt{}`good music'
\end{exe}

\subsection{Determiners}
Thiguka has proximal, medial, and distal determiners. Determiners are placed before root words. 

\begin{exe}
    \ex{} \gll{}ka\textapprox{}gu-parthas \emph{kasa} kaela\\
    \agradj{}-pink.purple PROX.DET tree\\
    \glt{}`\emph{this} sakura tree'
\end{exe}

\begin{exe}
    \ex{} \gll{}ka\textapprox{}gu-parthas \emph{saka} kaela\\
    \agradj{}-pink.purple MED.DET tree\\
    \glt{}`\emph{that} sakura tree \emph{near you}'
\end{exe}

\begin{exe}
    \ex{} \gll{}ka\textapprox{}gu-parthas \emph{asila} kaela\\
    \agradj{}-pink.purple DIS.DET tree\\
    \glt{}`\emph{that} sakura tree'
\end{exe}

\subsection{Tense-aspect}
Thiguka has a tense-aspect system.

\begin{table}[h!]
    \centering
    \caption{Tense}
    \begin{tabularx}{8cm}{|X|X|}
        \hline
        \textbf{Tense} & \textbf{Suffix} \\
        \hline
        Past & -ta \\
        Recent past & -tu \\
        Present & -sa \\
        Future & -tha \\
        \hline
    \end{tabularx}
\end{table}

\begin{table}[h!]
    \centering
    \caption{Aspect}
    \begin{tabularx}{8cm}{|X|X|}
        \hline
        \textbf{Aspect} & \textbf{Prefix} \\
        \hline
        Perfective & pa- \\
        Perfect & fe- \\
        Progressive & ge- \\
        Imperfective & sa- \\
        Contemplative & kas- \\
        Accidental & ras- \\
        Intentional & ka- \\
        \hline
    \end{tabularx}
\end{table}

\subsection{Pronouns}
Thiguka completely lacks gender distinction in pronouns, and has a clusivity distinction in first person pronouns.

\begin{table}[h!]
    \centering
    \caption{Pronouns}
    \begin{tabularx}{15cm}{|X|X|X|X|}
        \hline
        person/number & \textbf{First} & \textbf{Second} & \textbf{Third} \\
        \hline
        singular & thaka & kake & pothu \\
        plural   & & katake & rafu \\
        inclusive plural & situ & & \\
        exclusive plural & pata & & \\
        \hline
    \end{tabularx}
\end{table}

Thiguka has no T-V distinction in pronouns. Politeness may be achieved by attaching positive adjectives to second-person pronouns, but this is discouraged and goes against Lemuria's intentions.

\subsection{Template}
Thiguka positions morphemes in a specific order, based on whether it is a noun or verb.

\subsubsection{Noun}
For nouns, the order is as follows: Derivational prefixes, Negation or intensifiers, Stem, Case, Adverb suffixes, Plurality.

\subsubsection{Verb}
For verbs, the order is as follows: Derivational prefixes, Negation or intensifiers, Aspect, Stem, Tense, Adverb suffixes

\subsection{Derivation}
Thiguka has multiple methods of taking in new words into its lexicon; such as compounding and the addition of derivational prefixes.

A few select examples of Thiguka derivational prefixes are:

\begin{enumerate}
    \item fah-, agent prefix akin to English -er
    \item ri-, intensifier
    \item li-, negation
    \item filay-, ``in the presence of a cat''
\end{enumerate}

\begin{exe}
    \ex{} \gll{}fah-katarila\\
    AGT-space.travel\\
    \glt{}`space traveler'
\end{exe}

\begin{exe}
    \ex{} \gll{}fah-filay-katarila\\
    AGT-with.cat-space.travel\\
    \glt{}`one who travels through space with a pet cat'
\end{exe}

\begin{exe}
    \ex{} \gll{}fah-filay-li-katarila\\
    AGT-with.cat-NEG-space.travel\\
    \glt{}`one who does not travel through space with a pet cat'
\end{exe}



\newpage
\section{Orthography}
Thiguka's orthography is phonetic, and is as follows.

\begin{table}[h!]
    \centering
    \caption{Thiguka orthography}
    \begin{tabularx}{4cm}{|X|X|}
        \hline
        \textbf{IPA} & \textbf{Letter} \\
        \hline
        p & p \\
        t & t \\
        k & k \\
        b & b \\
        d & d \\
        g & g \\
        f & f \\
        s & s \\
        r & r \\
        l & l \\
        \textipa{T} & th \\
        a & a \\
        e & e \\
        i & i \\
        o & o \\
        u & u \\
        \textipa{A} & ah \\
        ai & ay \\
        ei & ey \\
        ui & uy \\
        au & aw \\
        \textipa{Ai} & auy \\
        ia & ija \\
        \hline
    \end{tabularx}
\end{table}

Thiguka follows English punctuation conventions.  

Loanwords are respelled to fit this orthography. The native spelling can be provided after the loanword in a parenthesis, for example:

\begin{quotation}
    Lela Reyl (Ilgis: Lena Raine) gulaya tagu-Arika tarifari.
\end{quotation}

\subsubsection{Proper nouns and hyphens}
Proper nouns are capitalized, but if they are being attached to a word --- treated as an adjective, the prefix that comes before the adjective is usually separated from the adjective by a hyphen: \emph{tagu-Ayris}, not \emph{*Taguayris} or \emph{*taguAyris}.

Writers also have the option of placing hyphens between proper nouns and affixes. \emph{Reyl-pah} and \emph{Reylpah} are both correct. However, this should usually not be done for proper nouns: \emph{*kaela-taygula} outside of glosses.


\newpage
\section{Lexicon}
Thiguka, as of May 2024, has about 350 defined words, with slight bias towards Internet and music-related terminology.

Thiguka's full lexicon is available at \url{https://github.com/thiguka/lexicon} in both tab-separated values format and an Anki deck.
A select few words from Thiguka's lexicon are as follows:

\subsection{Swadesh-Yakhontov list}
Because of the author's laziness and some holes in the lexicon, he is only able to provide a 35-word Swadesh-Yakhontov list below:

\begin{multicols}{2} 
\begin{enumerate}
    \item I --- thaka
    \item you (singular) --- kake
    \item this --- kasa
    \item who --- kalisu pelu
    \item what --- kalisu
    \item one --- isa 
    \item two --- dalata
    \item fish --- gothi
    \item dog --- aso
    \item louse --- lafauy
    \item blood --- rahsa
    \item bone --- bole
    \item egg --- logi
    \item horn --- kipi
    \item tail --- taysa
    \item ear --- eyla
    \item eye --- fathas
    \item nose --- fothes
    \item tooth --- tutho
    \item tongue --- kageta
    \item hand --- aysi
    \item know --- kala
    \item die --- dasay
    \item give --- salath
    \item sun --- asalisa
    \item moon --- luti
    \item water --- tubi
    \item salt --- salu
    \item stone --- sute
    \item wind --- lufah
    \item fire --- apay
    \item year --- fasate
    \item current year --- saro
    \item full --- hakath
    \item new --- bage
    \item name --- leyle
\end{enumerate}
\end{multicols}
\newpage

\section{Syntax}
\subsection{Word order}
Thiguka always uses a subject-verb-object word order.
While hypothetically it may be possible to use alternate word orders due to Thiguka case marking, SVO is the intended order and alternative orders will cause possible confusion.

\subsection{Questions}
Two primary strategies for asking questions in Thiguka are to create declarative statements about the listener and allow the listener to fill in the blanks.

The intonation of these questions is rising; similar to English.

\begin{exe}
    \ex{} \gll{}Kake-pafay leyle las\\
    2SG-GEN name COP\\
    \glt{}`Your name is?'
\end{exe}

Thiguka has one single determiner, `kalisu' (what?).

A noun can be placed after `kalisu' to further form the other English wh-words.

\begin{table}[h!]
    \centering
    \caption{Translations of English wh-words}
    \begin{tabularx}{8cm}{|X|X|}
        \hline
        \textbf{Thiguka} & \textbf{English wh-word} \\
        \hline
        kalisu & what \\
        kalisu pelu & who \\
        kathisa & why \\
        kalisu pasla & where \\
        kalisu tayle & when \\
        \hline
    \end{tabularx}
\end{table}

`kathisa' is a shortened form of `kalisu thilpahelah kufa kaketay'.

\begin{exe}
    \ex{} \gll{}Kalisu thil-pah-elah kufa kake-tay?\\
    What thing-NOM-PL coerce 2SG-ACC?\\
    \glt{}`What made you do it?'
\end{exe}

An example of `kalisu' is as follows:

\begin{exe}
    \ex{} \gll{}Kalisu las kakepafay leyle?\\
    what COP 2SG-GEN name\\
    \glt{}`What is your name?'
\end{exe}

\subsubsection{Polar}
Polar questions are marked by the particle \emph{kalu} added to the bottom of a word.

\begin{exe}
    \ex{} \gll{}Kake las pe~gu-gulid pelu kalu?\\
    2SG-NOM COP \agradj{} person POLAR \\
    \glt{}`Are you a good person?'
\end{exe}

\subsection{Posession}
Posession is conveyed using the genitive case suffix \emph{-pafay}.

\begin{exe}
    \ex{} \gll{}Thaka-pafay kaela-gula\\
    1SG-GEN tree-DU\\
    \glt{}`My two trees'
\end{exe}

\subsection{Negation}
Negation is conveyed with the morpheme \emph{li}, which can either be a prefix or a particle.

Negation cannot negate negation.
Adding more negation only strengthens the negation instead of canceling it out.

\begin{exe}
    \ex{} \gll{}Li! Thaka-pah li kisu-tha pothu-tay\\
    No! 1SG-NOM NEG kill-FUT 3SG-ACC\\
    \glt{}`No! I will not kill them!'
\end{exe}

\begin{exe}
    \ex{} \gll{}Li-thaka-pah kisu-ta pothu-tay,\\
    NEG-1SG-NOM kill-PST 3SG-ACC\\
    \glt{}`No, I didn't kill them.'
    \glt{}lit. `Not-me killed them.'
\end{exe}

\begin{exe}
    \ex{} \gll{}Li-li-li-li-thaka-pah kisu-ta pothu-tay,\\
    NEG-NEG-NEG-NEG-1SG-NOM kill-PST 3SG-ACC\\
    \glt{}`I definitely did not kill them.'
\end{exe}

\subsection{Relative clauses}
Thiguka does not yet have a relative clause system. 
Translations created before the creation of the system attached information conveyed by the relative clause as adjectives, or simply created additional declarative sentences.

\begin{exe}
    \ex{} \gll{}Rey~gu-teraseleter Reyl gulaya ta~gu-ri-gulid tarifari.\\
    AGR~ADJ-transgender Raine COP.3SG.PRS AGR~ADJ-INT-good musician\\
    \glt{}`Raine, who is transgender, is a great musician.'
    \glt{}lit. `transgender-Raine is a great musician.'
    
\end{exe}

\newpage
\section{Examples}
A variety of texts have been translated into Thiguka. The first major complete translation, from mid-April 2024, is Wikipedia's article on Lena Raine, an American music composer from Seattle who has made music for Minecraft.

The translation on Lena Raine is subject to errors, being an early text and not having been updated to reflect the latest changes to the Thiguka language as of May 2024.

\subsection{Thiguka's endonym}
\ex
\begingl
    \gla \textbf{thi}~\textbf{gu}-\textbf{ka}geta thil//
    \glb \agradj{}-mouth thing//
    \glft `mouth thing'//
\endgl
\xe

\subsection{Miscellaneous sentences}

\ex
\begingl
    \gla Lahkelakesi-pah fusala laki!//
    \glb make.building-NOM must grow//
    \glft `The factory must grow!'//
\endgl
\xe

\ex
\begingl
    \gla Kalisu pelu pa-gasari-ta?//
    \glb what person PFV-action.V-PST//
    \glft `Who did it?'//
\endgl
\xe

\ex
\begingl
    \gla Kake-pah fusala lahkela lahfisipaka-tay-sa.//
    \glb 2SG-NOM must make conlang-ACC-SG//
    \glft `You must make a conlang.'//
\endgl
\xe

\chapter{Lena Raine}
This article on Lena Raine comes from Wikipedia. Wikipedia's content is licensed under CC BY-SA 4.0; the text of this license is at \url{https://en.wikipedia.org/wiki/Wikipedia:Text_of_the_Creative_Commons_Attribution-ShareAlike_4.0_International_License}. A verbatim copy of the original English article text is included here for easier comparison.

\section{Thiguka}
Lela Reyl (Ilgis: Lena Raine; berethata 1984-02-29), leylesathasi Lela Sapel (Ilgis: Lena Chappelle) aluthasi Kureyl (Ilgis: Kuraine) gulaya Taguarika alu Tagukalada tagulikageta tarifari alu lisapahsa lahkelasa kolugeyletayelah.
Pelupahelah kalasa Reyltay pur pothupafay leyguritu leyboro ifil Selesete alu Maylakraf alu kigugiled kisufetelah dalata.
Pothupah lahkelasathasi rifaritay ifil kolugeyletayelah sufadhas Deletarule alu Sikori: Sigukothor Sifothisa. 

\subsection{Siguaga sifata}
Reylpah berethata 1984-02-29 ifil Siatel ifil Asiltol.
Pothupafay tali tarifari alu pothupafay laysa pelifefi.
Fipothupafay tagukuyri tarifagetata alu pothupah fathusita pothutay ifil tarifari.
Pothupafay tali-thasi fahgahlirithi.
Solik Edeog sigupeke pegusifothi peli giledsa fathusita pothutay ifil pagulidi patefarielah.
Reylpah pagulidi patefarikithielah taguisasafi lahkeluta rifaritayelah pothu kalata aludhete lahkelata pothupafay tagubage tarifari.
Kuthifafay, pothupah kalata ifil Kornis ogukala ose Arafatelah.
Kuthifafay, Kornispah salathta Reyltay lahgurifari lahkelasa oselufasa.

\subsection{Sifutila}
Pelupahelah kalasa Reyltay pur pothupafay leyguritu leyboro ifil Selesete alu Kigugiled Kisufetelah Dalata. 
Pothupah leyborota ifil Kigugiled Kisufetelah Dalata ArilaLetkala pur fagugati fasatelah gafalu fahpaterelahkesa alu fahlahkelapahsa ritutay.

Pothupah alu Lakleyl Diler, gafalu oguifil ose fahlahkelapahgula ritutay ifil geylepafay lisufasa ifil 2015, Kigugiled Kisufetelah Dalata: Arfaspafaysa Thigusapagad Thilelah.
Pothupah falgathata ArilaLetay ifil 2016, bulu pogufilasar pothupah lahkelata rifaritayelah ifil ArilaLetpafay kolugeylelah.

Ifil 2018, Reylpah fathusita afegurayfathil koguafeltur kolugeyle esektay ifil itisayokala.
Reylpah lahkelata rifaritay alu lahkelata kolugeyletay esektay alu Deytaireyspah (Ilgis: Dataerase) lahkelata arafatayelah.
Ifil 2019, pothupah fathusita riguisasafi rifari, Isakalasa.

Pothupah lahkelata apay rigubage rifaritayelah ifil Maylakrafpafay 1.16 "Lether Patifekalasa" (Ilgis: Nether Update) ifil 2020.
Faguisa fasatefay, pothupah alu tagukolugeyle tarifarisa Kuli Taliokapah lahkelata gati rigubage rifaritayelah ifil 1.18 "Kelefitelah alu Sahlagelah: Pagudalatafasi Parafsa" patifekalasa, Faguisa fasatefay, pothupah lahkelata talo rigubage rifaritayelah ifil 1.19 "Pagufaylata Parafsa".

Reylpah lahkelata rifaritay ifil araguafeltur koguarapigi kolugeylekala Sikori: Sigukothor Sifothisa alu tulorta pelutayelah lahkelata rifaritayelah ifil sagudalatafasi sagdathasa ifil Deletarunekala.
Kolugeylepahgula fathusita ifil 2021.

\ex
\begingl
\gla Ifil 2024-05-28, Reyle-pah pa-fathusi-ta      kolugeyle  Alotherili-pafay esahla    kala-tay-sa.//
\glb  on   2024-05-28, Raine-NOM PFV-distribute-PST video.game Anothereal-GEN   existence knowledge-ACC-SG.//
\glft On May 28, 2024, Raine distributed knowledge of Anothereal's existence.//
\endgl
\xe

\ex
\begingl
\gla Alotherili-pah pafay ko\~{}gu-sutebup-tay       alu ko\~{}gu-tugasa     thi\~{}gu-kolugeyle   thil-tay-elah.//
\glb Anothereal-NOM has   AGR\~{}ADJ-shoot.em.up-ACC and AGR\~{}ADJ-roleplay AGR\~{}ADJ-video.game thing-ACC-PL//
\glft Anothereal has shoot 'em up and roleplay game things.//
\endgl
\xe

\subsection{Sigupelu sifata}
Reyguteraseleter Reylpahthasi rayfata agladtaysa, Siti of Taygers aluthasi rayfasa siguelegibiti sigupeke sifothi ifil pothupafay taygurifeyla tayle.

\subsection{Thiguolota thilelah}
Asagata Reyltay pur ologubafata thiguolota thilsa aluthasi pothupafay rifari ifil Selesete risifta Giguarika Giledpafay Tarifarielah, pelupahelah rayfasa agladtayelah alu pelupahelah fathusisa thiltayelah rigulid\\ rigukolugeyle rifari ifil fasate thiguolota thilsa ifil 2019.
Pothupafay rifari ifil Selesetepahthasi risifta rigurigulid asagurifari asagasa ifil thigukolugeyle thiguolota thilelah ifil 2018.

\section{Translation notes}
As Lemuria works on Thiguka, his knowledge of linguistics grows ever-larger.
Here is a translation of the first sentence of Wikipedia's article on Lena Raine, in two versions; the old version from April 2024 and the new one from June 2024.

\subsubsection*{April 2024}

\ex
\begingl
\gla Lela Reyl   (beretha-ta 1984-02-29),   leyle-sa-thasi Lela Sapel      alu-thasi Kureyl gulaya       Ta\~{}gu-arika     alu Ta\~{}gu-kalada   ta\~{}gu-li-kageta   tarifari alu lah\~{}gu-lisa   fah-lahkela-pah kolugeyle-tay-elah.//
\glb Lena Raine  (born.V-PST 1984-02-29)    name-PST-also  Lena Chappelle  and-also  Kuraine COP.3SG.PRS AGR\~{}ADJ-America and AGR\~{}ADJ-Canada AGR\~{}ADJ-NEG-mouth musician and AGR\~{}ADJ-woman AGT-make-NOM    video.game-ACC-PL.//
\glft Lena Raine  (born Februrary 29, 1984), also known as Lena Chappelle and Kuraine, is an American and Canadian instrumental musician and woman making video games.//
\endgl
\xe

\newpage

\subsubsection*{June 2024}

\ex
\begingl
\gla Lela Reyl  (beretha-ta 1984-02-29), pafay-sa=thasi leyle Lela Sapel     o  Kureyl, gulaya      fah\~{}gu-Arika    alu fah\~{}gu-Kalada  fahlahpeyrifari-tay-sa alu fah\~{}gu-kolugeyle   fah-lahkela.//
\glb Lena Raine (born.V-PST 1984-02-29), have-PRS=also  name  Lena Chappelle or Kuraine COP.3SG.PRS AGR\~{}ADJ-America and AGR\~{}ADJ-Canada composer-ACC-SG        and AGR\~{}ADJ-video.game AGT-make//
\glft Lena Raine (born February 29, 1984), who has also the name Lena Chappelle or Kuraine, is an \\American and Canadian composer, and video game maker.//
\endgl
\xe

\section{Gloss}

\ex
\begingl
\gla  Lela Reyl (beretha-ta 1984-02-29), pafay-sa=thasi leyle Lela Sapel o Kureyl, gulaya fah~gu-Arika alu fah~gu-Kalada fahlahpeyrifari-tay-sa alu fah~gu-kolugeyle fah-lahkela.//
\glb  Lena Raine (born.V-PST 1984-02-29), have-PRS=also name Lena Chappelle or Kuraine COP.3SG.PRS AGR~ADJ-America and AGR~ADJ-Canada composer-ACC-SG and AGR~ADJ-video.game AGT-make//
\glft Lena Raine (born February 29, 1984), who has also the name Lena Chappelle or Kuraine, is an American and Canadian composer, and video game maker.//
\endgl 
\xe

\ex
\begingl 
\gla  Pelu-pah-elah kala-sa Reyl-tay pur pothu-pafay ley~gu-ritu leyboro ifil Selesete alu Maylakraf alu ki~gu-giled kisufete-lah dalata.//
\glb  Person-NOM-PL know-PRS Raine-ACC for 3SG-GEN AGR~ADJ-sound work in Celeste and Minecraft and AGR~ADJ-guild war-PL two.//
\glft People know Raine for her soundtrack work in Celeste, {} Minecraft, and Guild Wars 2.//
\endgl 
\xe

\ex
\begingl
\gla  Pothu-pah lahkela-sa-thasi rifari-tay-elah ifil kolugeyle-tay-elah sufadhas Deletarule alu Sikori: Si~gu-kothor Sifothi-sa.//
\glb  3SG-NOM make-PST-also music-ACC-PL in video game-ACC-PL such as Deltarune and Chicory: AGR~ADJ-color tale-SG//
\glft She has also made music for other video games such as Deltarune and Chicory: A Colorful Tale.//
\endgl 
\xe

\subsection{Early life / Siguaga sifata}

\ex
\begingl
\gla  Reyl-pah beretha-ta  1984-02-29  Siatel-kala Asiltol-kala.//
\glb  Raine-NOM born-PST    Seattle-LOC  Washington-LOC.//
\glft Raine was born on February 29, 1984 in Seattle, Washington.//
\endgl
\xe

\ex
\begingl
\gla  Pothu-pafay tali  tarifari alu pothu-pafay laysa  pelifefi.//
\glb  3SG:N-POS father musician and 3SG:N-POS mother dancer.//
\glft Her father is a musician and her mother is a dancer.//
\endgl
\xe

\ex
\begingl
\gla  Fi-pothu-pafay ta~gu-kuyri tarifageta-ta alu pothu-pah fathusi-ta pothu-tay ifil tarifari.//
\glb  DIM-3SG:N-POS AGR~ADJ-choir sing-PST and 3SG-NOM exposed 3SG-ACC in music.//
\glft Younger-her, in a choir, sang and that exposed her to music.//
\endgl
\xe

\ex
\begingl
\gla  Pothu-pafay tali-thasi  fah-gahlirithi.//
\glb  Her-POS father-also AGT-violin.//
\glft Her father-also was a violinist.//
\endgl
\xe

\ex
\begingl
\gla  Solik Edeog si~gu-peke pe~gu-sifothi peli giled-sa fathusi-ta pothu-tay ifil pa~gu-lidi patefari-elah.//
\glb  Sonic the Hedgehog AGR~ADJ-fictional AGR~ADJ-story-PL community-NOM-SG expose-PST 3SG-ACC in AGR~ADJ-midi arrangement-PL.//
\glft A Sonic the Hedgehog fan community exposed her to MIDI arrangements.//
\endgl
\xe

\ex
\begingl
\gla  Reyl-pah pa~gu-lidi patefari-kithi-elah ta~gu-isa-safi lahkelu-ta rifari-tay-elah pothu kala-ta alu-dhete lahkela-ta pothu-pafay ta~gu-bage tarifari.//
\glb  Raine-NOM AGR~ADJ-midi arrangement-INS-PL AGR~ADJ-one-ORD remake-PST song-ACC-PL 3SG know-PST and-then make-PST 3SG-POS AGR~ADJ-new music.//
\glft Raine, using MIDI arrangements, first recreated songs she knew and then made her own original music.//
\endgl
\xe

\ex
\begingl
\gla  Kuthifa-fay, pothu-pah kala-ta  Kornis o~gu-kala ose-kala  Arafat-elah.//
\glb  long time.after-ADV, 3SG-NOM study-PST Cornish AGR~ADJ-knowledge house-LOC Art-PL.//
\glft Later, she studied in Cornish College of the Arts.//
\endgl
\xe

\ex
\begingl
\gla  Kuthifa-fay, Kornis-pah salath-ta Reyl-tay  lah~gu-rifari lahkela-sa oselufa-sa.//
\glb  long time.after-ADV, Cornish-NOM give-PST Raine-NOM  AGR~ADJ-music make-PST degree-SG//
\glft Later, Cornish gave Raine a music composition degree.//
\endgl
\xe

\subsection{Career}
\ex
\begingl
\gla Pelu-pah-elah kala-sa Reyl-tay pur pothu-pafay ley~gu-ritu leyboro ifil Selesete alu Ki~gu-giled Kisufete-lah Dalata. //
\glb person-NOM-PL know-PRS Raine-ACC for 3SG-POS AGR~ADJ-sound work in Celeste and AGR~ADJ-guild war-PL two. //
\glft People know Raine for her soundtrack work in Celeste and Guild Wars 2. //
\endgl
\xe

\ex
\begingl
\gla Pothu-pah leyboro-ta ifil Ki~gu-giled Kisufete-lah Dalata  ArilaLet-kala pur fa~gu-gati fasate-lah gafalu  fah-paterelahke-sa alu fah-lahkela-pah-sa  ritu-tay. //
\glb 3SG-NOM work-PST in AGR~ADJ-guild war-PL two  ArenaNet-LOC for AGR~ADJ-six year-PL as  AGT-design-SG and AGT-make-NOM-SG  sound-ACC. //
\glft She worked on Guild Wars 2 at ArenaNet for six years as a designer and composer of soundtracks. //
\endgl
\xe

\ex
\begingl
\gla Pothu-pah alu Lakleyl Diler, gafalu o~gu-ifil ose fah-lahkela-pah-gula  ritu-tay ifil  geyle-pafay lisufa-sa ifil 2015, Ki~gu-giled Kisufete-lah Dalata: Arfas-pafay-sa  Thi~gu-sapagad Thil-elah. //
\glb 3SG-NOM and Maclaine Diemer, as AGR~ADJ-in house AGT~make-NOM-DU  sound-ACC in  game-POS expand-SG in 2015, AGR~ADJ-guild war-PL two: heart-POS-SG  AGR~ADJ-sharp thing-PL. //
\glft She and Maclaine Diemer, as in-house composers of soundtracks in the game's expansion in 2015, Guild Wars 2: Heart of Thorns. //
\endgl
\xe

\ex
\begingl
\gla Pothu-pah falgathata ArilaLet-ay ifil 2016, bulu po~gu-filasar pothu-pah lahkela-ta rifari-tay-elah ifil ArilaLet-pafay   kolugeyle-lah. //
\glb 3SG-NOM leave-PST ArenaNet-ACC in 2016, but AGR~ADJ-freelance 3SG-NOM make-PST music-ACC-PL in ArenaNet-POS  video game-PL //
\glft She left ArenaNet in 2016, but freelance her made songs for ArenaNet's other video games. //
\endgl
\xe

\ex
\begingl
\gla Ifil 2018, Reyl-pah fathusi-ta  afe~gu-rayfathil ko~gu-afeltur kolugeyle esek-tay ifil itisayo-kala. //
\glb In 2018, Raine-NOM release-PST  AGR~ADJ-written.thing AGR~ADJ-adventure game ESC-ACC in itch.io-LOC. //
\glft In 2018, Raine released the text adventure video game ESC on itch.io. //
\endgl
\xe

\ex
\begingl
\gla Reyl-pah lahkela-ta rifari-tay alu lahkela-ta kolugeyle-tay esek-tay alu Deytaireys-pah lahkela-ta arafat-ay-elah. //
\glb Raine-NOM make-PST music-ACC and make-PST video game-ACC ESC-ACC and Dataerase-NOM make-PST art-ACC. //
\glft Raine composed and developed ESC and Dataerase made visuals. //
\endgl
\xe

\ex
\begingl
\gla Ifil 2019, pothu-pah fathusi-ta  ri~gu-isa-safi rifari, Isa-kala-sa. //
\glb In 2019, 3SG-NOM release-PST  AGR~ADJ-one-ORD music, one-know-PST. //
\glft In 2019, she released her first album, Oneknowing. //
\endgl
\xe

\ex
\begingl
\gla Pothu-pah lahkela-ta apay ri~gu-bage rifari-tay-elah ifil Maylakraf-pafay 1.16 "Lether Patife-kala-sa" ifil 2020. //
\glb 3SG-NOM make-PST four AGR~ADJ-new music-ACC-PL in Minecraft-POS 1.16 "Nether change-LOC-SG" in 2020. //
\glft She composed four new songs for Minecraft's 1.16 "Nether Update" in 2020. //
\endgl
\xe

\ex
\begingl
\gla Fa~gu-isa fasate-fay, pothu-pah alu  ta~gu-kolugeyle tarifari-sa Kuli Talioka-pah lahkela-ta gati ri~gu-bage rifari-tay-elah ifil 1.18 "Kelefitelah alu Sahlagelah: Pa~gu-dalata-fasi Paraf-sa" patife-kala-sa, //
\glb AGR~ADJ-one year-after.ADV, 3SG-NOM and  AGR~ADJ-video game composer-SG Kumi Tanioka-NOM make-PST six AGR~ADJ-new music-ACC-PL in 1.18 cave-PL and cliff-PL AGR~ADJ-two-ORD part-SG" change-LOC-SG. //
\glft After one year, she and fellow video game composer Kumi Tanioka  made six new songs for 1.18 "Caves and Cliffs: Second Part" update. //
\endgl
\xe

\ex
\begingl
\gla Fa~gu-isa fasate-fay, pothu-pah lahkela-ta talo ri~gu-bage rifari-tay-elah ifil 1.19 "Pa~gu-faylata Paraf-sa". //
\glb AGR~ADJ-one year-after.ADV, 3SG-NOM make-PST three AGR~ADJ-new music-ACC-PL for 1.19 AGR~ADJ-wild change-SG. //
\glft After one year, Raine made three new songs for 1.19 "Wild Update". //
\endgl
\xe

\ex
\begingl
\gla Reyl-pah lahkela-ta rifari-tay ifil ara~gu-afeltur ko~gu-arapigi kolugeyle-kala Sikori: Si~gu-kothor Sifothi-sa alu tulor-ta pelu-tay-elah lahkela-ta rifari-tay-elah ifil sa~gu-dalata-fasi sagdatha-sa ifil Deletarune-kala. //
\glb Raine make-ACC music for AGR~ADJ-adventure AGR~ADJ-rpg video game-LOC Chicory: AGR~ADJ-color tale-SG and assist-PST person-ACC-PL make-PST music-ACC-PL in AGR~ADJ-two-ORD chapter-SG ifil Deltarune-LOC. //
\glft Raine created music for adventure RPG game Chicory: Colorful Tale and assisted people making music in second chapter of Deltarune. //
\endgl
\xe

\ex
\begingl
\gla Kolugeyle-pah-gula fathusi-ta ifil 2021. //
\glb video game-NOM-DU release-PST in 2021. //
\glft Both games released in 2021. //
\endgl
\xe

\ex
\begingl
\gla Ifil 2024-05-28, Reyle-pah pa-fathusi-ta kolugeyle Alotherili-pafay esahla kala-tay-sa. //
\glb on 2024-05-28, Raine-NOM PFV-distribute-PST video.game Anothereal-GEN existence knowledge-ACC-SG. //
\glft On May 28, 2024, Raine distributed knowledge of Anothereal's existence. //
\endgl
\xe

\ex
\begingl
\gla Alotherili-pah pafay ko~gu-sutebup-tay alu ko~gu-tugasa thi~gu-kolugeyle thil-tay-elah. //
\glb Anothereal-NOM has AGR~ADJ-shoot.em.up-ACC and AGR~ADJ-roleplay AGR~ADJ-video.game thing-ACC-PL. //
\glft Anothereal has shoot 'em up and roleplay game things. //
\endgl
\xe

\subsection{Personal life}

\ex
\begingl
\gla  Rey~gu-teraseleter Reyl-pah-thasi rayfa-ta aglad-tay-sa, Siti of Taygers alu-thasi rayfa-sa si~gu-elegibiti si~gu-peke sifothi ifil pothu-pafay tay~gu-rifeyla tayle.//
\glb  AGR~ADJ-transgender Raine-NOM-also write-PST book-ACC-SG,  City of Tigers and-also write-PRS AGR~ADJ-lgbt AGR~ADJ-fake story in 3SG-POS AGR~ADJ-remaining time.//
\glft transgender Raine also wrote a book titled City of Tigers and also writes LGBT-oriented fictional stories in her spare time.//
\endgl
\xe

Thiguka has no construction for "is", so to communicate that Raine is transgender, I had to attach it to Raine's name as an adjective.

\subsection{Awards}

\ex
\begingl
\gla  Asaga-ta Reyl-tay pur olo~gu-bafata thi~gu-olota thil-sa alu-thasi pothu-pafay rifari ifil Selesete risif-ta Gi~gu-arika Giled-pafay Tarifari-elah, pelu-pah-elah rayfa-sa aglad-tay-elah alu pelu-pah-elah fathusi-sa thil-tay-elah ri-gulid ri~gu-kolugeyle rifari ifil fasate thi~gu-olota thil-sa ifil 2019.//
\glb  nominate-PST Raine-ACC for AGR~ADJ-bafta AGR~ADJ-honor thing-SG and 3SG-POS music in Celeste receive-PST AGR~ADJ-America society-POS  musician-PL, person-PL write-PRS book-NOM-PL and person-PL expose-PRS thing-NOM-PL INT-good AGR~ADJ-video game music in year AGR~ADJ-honor thing-SG in 2019//
\glft nominated Raine for a BAFTA award and also her music in Celeste received American Society of Musicians, Authors and Publishers best video game music of the year award in 2019.//
\endgl
\xe

\ex
\begingl
\gla  Pothu-pafay rifari ifil Selesete-pah.thasi risif-ta ri~gu-ri-gulid asa~gu-rifari asaga-sa ifil  thi~gu-kolugeyle thi~gu-olota thil-elah ifil 2018//
\glb  3SG-POS music in Celeste-NOM.also receive-PST AGR~ADJ-INT-good AGR~ADJ-music nomination-SG in  AGR~ADJ-video game AGR~ADJ-honor thing-PL in 2018//
\glft Her music in Celeste also received Best Music nomination in The Game Awards 2018.//
\endgl
\xe

\newpage

\section{English}
Lena Raine (/ˈleɪ.nə/ LAY-nə or /ˈlɛ.nə/ LEN-ə; born February 29, 1984), also known as Lena Chappelle or Kuraine, is an American-Canadian composer, producer, and video game developer.
Raine is best known for her work on the soundtracks of Celeste, Minecraft and Guild Wars 2.
She has composed music for various other video games, including Deltarune and Chicory: A Colorful Tale.

\subsection{Early life}

Raine was born on February 29, 1984 in Seattle, Washington.
Her father is a musician and her mother is a dancer.
She had an early exposure to music due to participation in choir at a young age.
Her father was also a violinist.
Through a Sonic the Hedgehog fan community she was introduced to MIDI arrangement, first recreating versions of songs she knew and then making original music.
She later attended Cornish College of the Arts for a degree in music composition.

\subsection{Career}

Raine is best known for her work on the soundtracks of Celeste and Guild Wars 2.
She worked on Guild Wars 2 at ArenaNet for six years, as a designer and soundtrack composer.
At ArenaNet, she and Maclaine Diemer were in-house composers of the music for the game's 2015 expansion, Guild Wars 2: Heart of Thorns.
She left ArenaNet in 2016, but has continued to occasionally compose songs for its various releases as a freelancer.

In 2018, Raine released the text adventure ESC on itch.io.
Raine was the developer and composer for ESC, with visuals created by Dataerase.
In 2019, she released her debut album, Oneknowing.
She composed music for Minecraft, creating four new pieces of music which were included in the 1.16 "Nether Update" in 2020.
A year later, she returned to Minecraft, composing six new tracks for the 1.18 "Caves \&{} Cliffs: Part II" update, alongside fellow video game composer Kumi Tanioka.
A year after that, she also composed another three new tracks for the 1.19 "The Wild Update", and then about two years later, she composed five new tracks for 1.21 "Tricky Trials".
Raine created the soundtrack for the adventure RPG Chicory: A Colorful Tale, and assisted with the soundtrack for the second chapter of Deltarune, both released in 2021.

On May 28, 2024, Raine announced Anothereal, a game combining elements of shoot 'em up and role-playing video games.

\subsection{Personal life}

Raine is transgender.
She has also written a book titled City Of Tigers, and has begun composing more LGBT-oriented fiction in her spare time.

\subsection{Awards}

Raine was nominated for a BAFTA (British Academy of Film and Television Arts) and won American Society of Composers, Authors and Publishers (ASCAP) Video Game Score of the Year in 2019 for Celeste. Her soundtrack for Celeste was also nominated for Best Score/Music at The Game Awards 2018.

\newpage

\subsection{Fun activities}
This was the sample passage for Conlang Ambassadors 3, a translation activity from May--June 2024.
\subsubsection*{English}
There are many forms of fun activities in general.
For example, there are hobbies which such as hiking, collecting themed objects, or non-professional creative arts and crafts.
Some people like to play games or sports, or just like to enjoy looking at people's pictures, listen to other people's music, or watch other people perform.
So long as someone enjoys the activity it is a reasonable recreational pastime.

\subsubsection*{Thiguka}
Elah lisure saparapahelah esahlasa.
Fulafatasa, pafaysa thigusapara thilelah, alu aragulisaleyfata arafat lores lisugufuros lilahkela ligupera lisure.
Pelupah falusaya geyletayelah, gasari isi fathasa lipothu pelupafay kurakatayelah, rififeyla lipothu pelutayelah, fathas pegulipothu pegusiligara pelutayelah, fathas pegulipothu pegusiligara pelutayelah.
Kula pelupah falusaya lisuretay, pothu gulaya ligurigulid lisure.

\subsubsection*{Gloss}
\begin{exe}
\ex{} \gll{}Elah lisure sapara-pah-elah esahlasa.\\
many fun.activity kind-NOM-PL exist.V-PRS\\
\glt{}`Many fun activity types exist.'
\end{exe}

\begin{exe}
\ex{} \gll{}Fulafata-sa, pafay-sa thi~gu-sapara thil-elah, alu ara~gu-li-saleyfata arafat-elah alu fulahthi lores lisu~gu-furos li-lahkela li~gu-pera lisure.\\
hike-PRS, have-PRS AGR~ADJ-type thing-PL, and AGR~ADJ-NEG-occupation art-PL and fun.make.thing-PL COP.3PL.PRS AGR~ADJ-fun NEG-make AGR~ADJ-money fun.activity.\\
\glt{}`Hiking, having themed objects, and non-professional art are fun not-money-making hobbies.'
\end{exe}

\begin{exe}
\ex{} \gll{}Pelu-pah falusaya geyle-tay-elah, gasari isi fatha-sa pe~gu-li-pothu pelu-pafay kuraka-tay-elah, rififeyla li-pothu pelu-tay-elah, fathas pe~gu-li-pothu pe~gu-siligara pelu-tay-elah.\\
 person-NOM like game-ACC-PL, action of see.V-PRS AGR~ADJ-NEG-3SG person-GEN image-ACC-PL, listen.to.music NEG-3SG person-ACC-PL, see.V AGR~ADJ-NEG-3SG AGR~ADJ-perform person-ACC-PL.\\
\glt{}`An unspecified amount of people like games, the act of seeing others' pictures, listening to others' music, or watching performing people.'
\end{exe}

\begin{exe}
    \ex{} \gll{}Kula pelu-pah   falusaya lisure-tay,       pothu gulaya      li~gu-ri-gulid   lisure.\\
                If   person-NOM enjoy    fun.activity-ACC, it    COP.3SG.PRS AGR~ADJ-INT-good fun.activity.\\
    \glt{}`If a person enjoys a fun activity, it is a good fun activity.'
\end{exe}




\chapter{C418}
To finish off all the Minecraft composers, here's C418.

\section{Thiguka}
Deylijel Roselfeldpah (berethata 1989-05-09), leylesathasi C418, gulaya tagu-Duylslald tarifari ifil kolugeyle Maylakraflay.
Pothu-tay pakalasa dasili pothupah gulaya fahguifiltatayle fahlahpeyrifaritaysa alu fahguritu fahlahkela ifil kolugeyle Maylakrafkala.
Pothu-pah pafay Deifi Redele-pafay alu Karla Sibolapafay talit, palahkelata gigulahkelakolugeyle giled Aifi Rod-tay.

Pothupah ferayfatathasi alu felahkelatathasi rifaritaysa Biold Stereysere Thillayelah alu Kuki Klikere-pafay fagu-Sti fathusilaysa.
Pothupah fediseytathasi Alabalagusi-lay, si gulaya rigu-Arika rigusutefari rifalagiled.

\subsection{Siguaga sifata}
Roselfeld-pah gulata paberethata ifil duguisefe Duylslald-kala ifil 1989-05-09.
Roselfeld-pah gulaya begu-Sofijet tagupaberethata alu taligu-Duylslald tagulitasafa alu tagufahlahgulelothi talipafay alu lagu-Duylslald laysapafay batoli.

Pothupah paparakalata gesalahrifaritay skiguaga Skisi Trakere-kithi, si gulaya klogu-Ilpuls Trakere klole, alu Eybeltolo Laif ifil tuguaga 2000-elah.
Rafupah poguta thigufikafahra thilgula ifil asila taylekala.
Pothu-pafay ibatoli, Ari Roselfeld-pah (Ilgis: Harry Rosenfeld) fathusi gesalahrifaritay pothulay Ilpuls Trakere-kithi.
Aripah pasipikta ``likalapeluifele'' (Ilgis: ``even an idiot'') lahgugulid lahkela rifari-tay pothukithi.
Pothupafay ibatoli, leylesathasi C818.
Asilapah pakufata pothutay pafay leyle C418, si gesipiksa asila leylepah gulaya lipafay sipithikala.

\newpage

\section{Gloss}

\ex
\begingl
          \gla  Deylijel Roselfeld-pah (beretha-ta 1989-05-09), leyle-sa=thasi C418,//
          \glb  Daniel Rosenfeld-NOM (born-PST 1989-05-09) name-PRS=also C418.PR//
          \glft Daniel Rosenfeld (born May 9, 1989) also named C418,//
\endgl
\xe

\ex
\begingl
          \gla  gulaya ta\~{}gu-Duylslald tarifari ifil kolugeyle Maylakraf-lay.//
          \glb  COP.3SG.PRS AGR\~{}ADJ-Germany musician in video.game Minecraft-DAT//
          \glft is a German musician in the video game Minecraft.//
\endgl
\xe

\ex
\begingl
          \gla  Pothu-tay pa-kala-sa dasili pothu-pah gulaya fah\~{}gu-ifil-tatayle fahlahpeyrifaritay-sa alu fah\~{}gu-ritu fah-lahkela ifil kolugeyle Maylakraf-kala//
          \glb  3SG-NOM PFV-know-PST because 3SG-NOM COP.3SG.PRS AGR\~{}ADJ-in-past composer-ACC-SG and AGR\~{}ADJ-sound AGT-make in video.game Minecraft-LOC//
          \glft He is known because he is the past composer and sound maker in the video game Minecraft.//
\endgl
\xe

\ex
\begingl
          \gla  Pothu-pah pafay Deifi Redele-pafay alu Karla Sibola-pafay talit, pa-lahkela-ta gi\~{}gu-lahkela-kolugeyle giled Aifi Rod-tay, si gulaya ge-lahkela-sa kolugeyle Alderstop-tay//
          \glb  3SG-NOM with Davey Wreden-GEN and Karla Zimonja-GEN help, PFV-make-PST AGR\~{}ADJ-make-video.game group Ivy Road.PR-ACC REL COP.3SG.PRS PROG-make-PRS video.game Wanderstop-ACC//
          \glft He, with Davey Wreden and Karla Zimonja's help, made the video game development organization Ivy Road, which is making Wanderstop.//
\endgl
\xe

\ex
\begingl
          \gla  Pothu-pah fe-rayfa-ta=thasi   alu fe-lahkela-ta=thasi rifari-tay-sa  Biold   Stereysere Thil-lay-elah alu Kuki   Klikere-pafay fa\~{}gu-Sti     fathusi-lay-sa  //
          \glb  3SG-NOM   PERF-write-PST=also and PERF-make-PST=also  music-ACC-SG   Beyond  Stranger   Thing-DAT-PL  and Cookie Clicker-GEN   AGR\~{}ADJ-Steam release-DAT-SG //
          \glft He has written also and made also music for Beyond Stranger Things, and Cookie Clicker's Steam release.//
\endgl
\xe

\ex
\begingl
          \gla  Pothu-pah fe-disey-ta=thasi Alabalagusi-lay, si  gulaya      ri\~{}gu-Arika     ri\~{}gu-sutefari     rifalagiled//
          \glb  3SG-NOM   PERF-dj-PST=also  Anamanaguchi-DAT REL COP.3SG.PRS AGR\~{}ADJ-America AGR\~{}ADJ-rock.music band//
          \glft He has DJed also for Anamanaguchi, which is an American rock band.//
\endgl
\xe

\newpage

\subsection{Early life / Siguaga sifata}
\ex
\begingl
\gla  Roselfeld-pah gulata      pa-beretha-ta ifil du\~{}gu-isefe Duylslald-kala ifil 1989-05-09.//
\glb  Rosenfeld-NOM COP.3SG.PST PFV-birth-PST  in   \agradj{}-east Germany-LOC    in   1985-05-09//
\glft Rosenfeld was born in East Germany on 1989-05-09,//
\endgl
\xe

\ex
\begingl
\gla  Roselfeld-pah gulaya      be\~{}gu-Sofijet ta\~{}gu-pa-beretha-ta alu tali\~{}gu-Duylslald ta\~{}gu-litasafa alu ta\~{}gu-fahlah-gulelothi tali-pafay      alu //
\glb  Rosenfeld-NOM COP.3SG.PST \agradj{}-Soviet \agradj{}-PFV-birth-PST and \agradj{}-Germany   \agradj{}-descent and \agradj{}-maker-gold      father-GEN.POSS and //
\glft Rosenfeld is a Soviet-born and German-descent and goldsmith father's and//
\endgl 
\xe


Descent was translated as \textit{litasafa}, the opposite of \textit{tasa} (``high, to go up, up'').

\ex
\begingl
\gla  la\~{}gu-Duylslald laysa-pafay batoli//
\glb  \agradj{}-Germany  mother-GEN.POSS son//
\glft German mother's son.//
\endgl
\xe

\ex
\begingl
\gla  Pothu-pah pa-parakala-ta gesa-lah-rifari-tay   ski\~{}gu-aga   skisi  trakere-kithi, si  gulaya      klo\~{}gu-Ilpuls  Trakere klole//
\glb  3SG-NOM   PFV-learn-PST  NMLZ-make-music-ACC \agradj{}-early Schism Tracker-INS,   REL COP.3SG.PRS \agradj{}-Impulse Tracker clone//
\glft He learned musicmaking with early Schism Tracker, which is an Impulse Tracker clone//
\endgl
\xe

\ex
\begingl
\gla  alu Eybeltolo Laif ifil tu\~{}gu-aga    2000-elah.//
\glb  and Ableton   Live in   \agradj{}-early 2000-PL//
\glft and Ableton Live, in the early 2000s.//
\endgl
\xe

\ex
\begingl
\gla  rafu-pah poguta      thi\~{}gu-fikafahra   thil-gula ifil asila tayle-kala//
\glb  3PL-NOM  COP.3DU.PST \agradj{}-rudimentary thing-DU  in   that  time-LOC//
\glft They were both rudimentary things in that time.//
\endgl
\xe

\ex
\begingl
\gla  Pothu-pafay  ibatoli, Ari   Roselfeld-pah, fathusi gesa-lah-rifari-tay   pothu-lay Ilpuls  Trakere-kithi//
\glb  3SG-GEN.POSS brother  Harry Rosenfeld-NOM  expose  NMLZ-make-music-ACC 3SG-DAT   Impulse Tracker-INS//
\glft It was his brother, Harry Rosenfeld, who introduced him to music composition through Impulse Tracker.//
\endgl
\xe

\ex
\begingl
\gla  Ari-pah   pa-sipik-ta   ``likalapelu=ifele'' lah\~{}gu-gulid    lahkela rifari-tay pothukithi.//
\glb  Harry-NOM PFV-speak-PST ``dumb.person=even`` \agradj{}-good.ADV make    music-ACC  3SG-INS//
\glft Harry said "even an idiot" can goodly create music with it.//
\endgl
\xe

"idiot" was translated as \textit{likalapelu} - "dumb person", lit. "not-knowledge person".

His brother was also known as C818, from which he chose the name C418, claiming that the name is "really cryptic and doesn't actually mean anything."

\ex
\begingl
\gla  Pothu-pafay  ibatoli, leyle-sa=thasi C818.//
\glb  3SG-GEN.POSS brother  name-PRS=also  C818.//
\glft His brother was also named C818.//
\endgl
\xe

\ex
\begingl
\gla  Asila-pah pa-kufa-ta   pothu-tay pafay leyle C418, si  ge-sipik-sa    asila leyle-pah gulaya      li-pafay sipithi-kala.//
\glb  that-NOM  PFV-make-PST 3SG-ACC   have  name  C418, REL PROG-speak-PRS that  name-NOM  COP.3SG.PRS NEG-have word-knowledge.//
\glft That made him have the name C418, saying that name is "really cryptic and has no meaning."//
\endgl
\xe

\section{English original (Wikipedia)}
Daniel Rosenfeld (born 9 May 1989), better known as C418 (pronounced "see four eighteen"),[4] is a German musician, producer and sound engineer.
He is best known as the former composer and sound designer for the sandbox video game Minecraft.
He is a co-founder of independent video game developer Ivy Road with Davey Wreden and Karla Zimonja, which is developing Wanderstop. 

He has also written and produced the theme for Beyond Stranger Things and the Steam release of Cookie Clicker.
He has also DJed for American rock band Anamanaguchi. 

\subsection{Early life}
Rosenfeld was born in East Germany on 9 May 1989, the son of a Soviet-born father of German descent working as a goldsmith and a German mother.
He learned to create music on early versions of Schism Tracker (a popular clone of Impulse Tracker) and Ableton Live in the early 2000s, both rudimentary tools at the time.
It was his brother, Harry Rosenfeld, who introduced him to music composition through Impulse Tracker, commenting that "even an idiot" can successfully create music with it.
His brother was also known as C818, from which he chose the name C418, claiming that the name is "really cryptic and doesn't actually mean anything."
% Rosenfeld has also stated that he was "mediocre at school," but learning basic music theory and English came easy to him.

\section{English literal}
Daniel Rosenfeld (born May 9, 1989) also named C418, is a German musician in the video game Minecraft.
He is known because he is the past composer and sound maker in the video game Minecraft.
He, with Davey Wreden and Karla Zimonja's help, made the video game development organization Ivy Road.

He has written also and made also music for Beyond Stranger Things, and Cookie Clicker's Steam release.
He has DJed also for Anamanaguchi, which is an American rock band.

\subsection{Early life}
Rosenfeld was born in East Germany on 1989-05-09.
Rosenfeld is a Soviet-born and German-descent and goldsmith father's and German mother's son.
He learned musicmaking with early Schism Tracker, which is an Impulse Tracker clone, and Ableton Live, in the early 2000s.
They were both rudimentary things in that time.
It was his brother, Harry Rosenfeld, who introduced him to music composition through Impulse Tracker.
Harry commented that "even an idiot" can goodly create music with it.
His brother was also named C818.
That made him have the name C418, saying that name is "really cryptic and has no meaning."


\section{Bee Movie script}

\subsection*{English}
According to all known laws of aviation, there is no way a bee should be able to fly.
Its wings are too small to get its fat little body off the ground.
The bee, of course, flies anyway because bees don't care what humans think is impossible.

\subsection*{Thiguka}
Lafad fagulufah lugufahofefi tharasagpahelah fiasaga bigulufahta bitaysa.
Pothupafay lirukpahelah parilitisi suf ofefi dogufalatha dogurilitisi doros okulusat gawralay.
Bipahsa lufahta alu likolaga pelupafayelah udiltayelah las asaga.

\subsection*{Gloss}
\ex
\begingl
\gla  Lafad fa~gu-lufah lu~gu-fahofefi tharasag-pah-elah fi-asaga bi~gu-lufah-ta bi-tay-sa.//
\glb  all AGR~ADJ-air AGR~ADJ-vehicle law-NOM-PL NEG-allow AGR~ADJ-fly-PRS bee-ACC-SG.//
\glft All airplane laws do not allow flying bees.//
\endgl
\xe

\ex
\begingl
\gla  Pothu-pafay liruk-pah-elah  pari-litisi suf ofefi  do~gu-falatha do~gu-rilitisi doros okulusat gawralay.//
\glb  3SG-POS wing-NOM-PL  excessive.INT-small to move  AGR~ADJ-fat AGR~ADJ-little body off ground//
\glft Its wings are too small to move a fat little body off the ground.//
\endgl
\xe

\ex
\begingl
\gla  Bi-pah-sa lufah-ta alu li-kolaga pelu-pafay-elah udil-tay-elah las asaga.//
\glb  Bee fly.V-PRS and NEG-comply person-GEN-PL think-ACC-PL COP allow.ADJ//
\glft Bees fly and do not comply with people's thoughts of what is allowed.//
\endgl
\xe





\end{document}
